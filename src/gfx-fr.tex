\chapter{GFX System}
Puisque le pipeline graphique du CPS-1 est câblé en dur dans le silicium des puces CPS-A et CPS-B, il n’y a aucun code à écrire ni rien à compiler.

\begin{figure}[H]
\sdraw{1.0}{gfx_arch}
\caption*{The GFX System components}
\end{figure}

Cela semble simple, mais il ne suffit pas de convertir les assets graphiques au format GFXROM.
Même si tous les types d’assets utilisent le même format de pixels, les lecteurs attentifs auront remarqué que les palettes ne sont pas stockées dans la GFXROM.
Il faut donc prendre soin de les sauvegarder séparément et de les fournir ultérieurement au processeur m68k, accompagnées d’un mécanisme permettant de les référencer.

Une autre contrainte importante concerne les "zones câblées en dur" de la GFXROM, dans lesquelles les assets des types OBJ, STAR1, STAR2, SCR1, SCR2 et SCR3 doivent obligatoirement être placés.
Si une tuile ne se trouve pas à l’endroit attendu, la commande de dessin correspondante est simplement ignorée.

Chaque carte utilise une disposition différente.
La carte de Street Fighter II, avec son PAL \icode{STF29}, divise les 6 Mio de la GFXROM en quatre zones : OBJ, SCR1, SCR2 et SCR3.

Une carte faisant tourner Final Fight utilise un PAL \icode{S224B}, qui découpe l’espace en ces quatre mêmes types de zones, mais avec des offsets différents et des proportions adaptées à un espace plus petit de 2 Mio.

Le PAL présent sur la carte de Forgotten Worlds, appelé \icode{LW621}, est plus complexe : il divise la GFXROM en cinq zones, afin d’y inclure également les bytecodes intercalés de STAR1 et STAR2.

\section{Format des tuiles}

Toutes les tuiles sont stockées de manière continue en mémoire, en utilisant des groupes de quatre octets codant huit pixels (ou "stylos").  
Les lignes de pixels sont stockées les unes à la suite des autres.  
La dimension d’une tuile varie en fonction des couches graphiques utilisées.  
OBJ et SCR2 utilisent les mêmes tuiles de **16x16**, ce qui signifie que chaque tuile occupe $16 \times 16 / 2 = 128$ octets.  
Pour **SCR1**, qui est plus finement réglé et utilise des tuiles de **8x8**, chaque tuile ne prend que **32 octets**.  
Enfin, le calque **SCR3**, plus grand, utilise **512 octets** par tuile de **32x32**.

\section{Organisation des GFX}

Les EPROM sont organisées en groupes de quatre puces, avec une sérialisation des données.  
Sur une carte comme celle de *Street Fighter II*, on trouve **trois groupes de quatre puces**.  
À l’intérieur de chaque groupe, les puces sont **entrelacées tous les WORD** (deux octets).

\nbdraw{gfx_interleave}

\section{Canaux}\label{channels}\index{Canaux!Pratique}

Les canaux mentionnés précédemment sont maintenant visibles.  
Les ROMs \icode{01} et \icode{02} sont appairées sur le **canal 1** pour fournir des données en **32 bits**.  
De même, les ROMs \icode{03} et \icode{04} sont associées sur le **canal 2**.  
La même division est appliquée aux autres ROMs selon le même principe.

\nbdraw{build_graph_gfx}

La carte de *Street Fighter II* produit un ensemble de **douze ROMs**.  
Chaque groupe stocke **2 Mio**.  
Le groupe [1--4] commence à l’adresse **0x000000**, le groupe [10--13] à **0x200000**, et les ROMs [20--23] à **0x400000**.  
Chaque puce doit être placée dans l’emplacement DIP portant le **même numéro** (voir page \pageref{boardb_no_chips}).










\section{À l’époque}\index{À l’époque!GFX}

Les éléments graphiques étaient la partie d’un jeu qui mobilisait le plus de personnes.  
Sur un grand titre comme *Street Fighter II*, cette tâche occupait à elle seule **vingt artistes**, dans une équipe de quarante personnes.

La création des tuiles de fond (tilemap) pour les couches **SCROLL** était confiée aux artistes juniors.  
Elle demandait peu de supervision, car ces décors devaient avoir une forme rectangulaire, et leur importance visuelle n’était pas capitale.

En revanche, la création des assets pour la couche **OBJ** (sprites) était beaucoup plus complexe.  
Elle se déroulait en **quatre étapes** : dessin des contours, allocation, dessin détaillé, puis "dotting" (pixelisation).

\subsection{Crayon et papier}

La direction artistique était décidée par le **planner** du jeu.  
Son objectif était de produire les concepts et les postures pour capturer l’essence d’un personnage, ainsi que de fixer les proportions et les mouvements.  
Ensuite, les artistes seniors prenaient le relais pour affiner le rendu.

\begin{q}{Eri Nakamura\cite{sf2devinterview}}
Quand j’ai rejoint l’équipe de *Street Fighter II*, Akiman avait déjà fait les esquisses préliminaires.

Nous étions quatre responsables des personnages : Satoru Yamashita, Yoshiaki Ohji, Ikuo Nakayama et moi.  
Satoru était le plus expérimenté, donc il a pris Ryu et Ken.

Un jour, Akiman est arrivé avec les croquis d’un catcheur professionnel, d’un sumo et d’une bête, et a dit :  
« Décidez qui fait quoi. »  
Pour être justes, on a fait **pierre-papier-ciseaux** pour décider !
\end{q}

Même si ce n’était pas courant, il arrivait parfois qu’un **planner prenne en charge l’ensemble du développement graphique d’un personnage**, surtout si celui-ci lui tenait à cœur.

\begin{q}{Akiman\cite{gameMaestro4}}
J’ai créé tous les graphismes de Chun-Li en seulement un mois.
\end{q}

Cela dit, les anecdotes comme celle de *Chun-Li* semblaient rester exceptionnelles.  
Par exemple, pour *Final Fight*, Satoru Yamashita a animé **Guy** et **Haggar**, mais **Akiman a dessiné les animations clés** de ces deux personnages.




\begin{figure}[H]
  \draw{ff_design}
  \caption*{Contours des sprites de Guy dans Final Fight, par Akiman}
  \end{figure}
  
  \subsection{Papier quadrillé non carré}\label{artists_par}
  
  Pour dessiner à la fois les contours et les versions détaillées des sprites, les artistes utilisaient un papier spécial avec un **système de double quadrillage**.
  
  Il y avait d’abord un **quadrillage léger**, dont les proportions n’étaient **pas carrées** afin de correspondre au **rapport d’aspect vidéo du CP-System**, soit **10:7**.  
  Grâce à cela, les artistes pouvaient dessiner normalement sans se soucier de la déformation de l’image, puisque les éléments rectangulaires correspondaient aux pixels étirés du CPS-1.
  
  Le papier comportait aussi un **deuxième quadrillage, plus foncé**, qui regroupait les cases en blocs de **16 par 16**, correspondant à la dimension des tuiles **OBJ**.  
  Ce quadrillage était essentiel pour l’étape d’**allocation**.
  
  \subsection{Allocation des OBJ}
  
  Si le fait de s’affranchir des sprites rectangulaires était une bénédiction pour les artistes, c’était un **casse-tête pour les chefs de projet** chez Capcom.  
  À une époque où les puces ROM étaient **très coûteuses**, chaque jeu se voyait attribuer un **budget ROM** au lancement, qu’il ne fallait **absolument pas dépasser**.
  
  Avant l’arrivée du CPS-1, respecter le budget était une simple affaire de division :  
  le nombre de sprites autorisés à l’équipe graphique était égal à **la taille de la ROM divisée par la taille d’un sprite rectangulaire**.  
  Mais avec des formes libres, un **nouveau problème de suivi** est apparu.
  
  \begin{figure}[H]
  \nbdraw{0x3300}
  \caption*{Feuille reconstituée de Dhalsim}
  \end{figure}
  
  La solution au problème d’allocation est venue… **avec du papier et des ciseaux**.
  
  \begin{q}{Nin\cite{sf2devinterview}}
  Pour tirer le meilleur parti de la capacité dont on disposait,  
  on écrivait la capacité totale de la ROM sur un panneau,  
  et on y collait les personnages pixelisés découpés sur papier.  
  S’il restait de la place sur le panneau, alors il restait de la capacité dans la ROM.  
  On remplissait les espaces comme dans un puzzle.
  
  On a gardé la création de la fin du jeu pour la fin du projet,  
  et à ce moment-là, on n’avait plus du tout de place.  
  On se demandait ce qu’on allait faire,  
  quand on a retrouvé **une planche qui avait glissé sous un bureau**.
  
  On l’a appelée la **"Mémoire miraculeuse"**.
  \end{q}
  



 \begin{figure}[H]
\img{rom_sheet_dhalsim.jpg}
\caption*{Feuille papier publiée de Dhalsim}
\end{figure}

Seules deux de ces feuilles ont été rendues publiques :  
l’une représente principalement **Dhalsim**\cite{ffdevinterview},  
et l’autre est connue sous le nom de **"feuille de Ryu"**\cite{htmcc}.  
Grâce à l’empreinte laissée dans la **GFXROM** et à la connaissance du format et de la disposition des pixels,  
**toutes les autres feuilles peuvent être reconstruites**.

Pour un jeu comme *Street Fighter II*, un budget de **6 Mio de GFX** a été validé.  
Avec **4,6 Mio alloués aux sprites**, **144 feuilles OBJ ont été imprimées**.  
Cela représentait **une quantité énorme à l’époque**, justifiée uniquement parce que l’équipe avait déjà réussi un gros coup avec *Final Fight*, qui tenait sur **un budget minuscule de 2 Mio**\cite{gameMaestro4}.

\begin{figure}[H]
\nbdraw{0x4500}
\caption*{Feuille reconstituée de Ryu}
\end{figure}

Comparer les documents publiés avec ce qui a réellement été livré dans le jeu est une source de **nombreuses découvertes et hypothèses**.

La feuille de **Dhalsim** se situe à l’offset \icode{0x3300} dans la GFXROM.  
Elle correspond presque parfaitement à la version papier, **à l’exception de la section commençant à \icode{0x60}**.  
L’une des poses a été supprimée au profit de l’animation de Chun-Li **"Hundred Rending Legs"**,  
ce qui laisse penser qu’il s’agissait d’un **ajout plus tardif**.

\begin{figure}[H]
\img{rom_sheet_ryu.jpg}
\caption*{Feuille papier publiée de Ryu}
\end{figure}

La feuille de Ryu, située à l’offset \icode{0x4500}, nous permet de **mieux deviner le processus de production**.  
De **grands sprites cohérents** montrent qu’au début du développement, plusieurs feuilles étaient allouées **par personnage**.  
Les tuiles étaient posées ensemble et regroupées autant que possible pour faciliter la **vérification visuelle**.

Mais à mesure que le projet avançait, l’équipe **a gratté jusqu’au moindre octet disponible**,  
et a commencé à allouer l’espace **tuile par tuile**.  
Il arrivait même qu’une **pose de personnage soit dispersée sur plusieurs feuilles**,  
comme sur celle de Dhalsim, où **des parties de Blanka** apparaissent.

\subsection{Le système de feuilles}

Mis à part quelques mentions éparses, **les employés de Capcom n’ont jamais expliqué en détail le fonctionnement du système de feuilles**.  
On ignore pour quels jeux et pendant combien de temps il a été utilisé au total.

Heureusement, la compréhension du format GFXROM **permet de remonter dans le temps**.  
La structure numérique est une **empreinte directe de ce à quoi ressemblaient les feuilles papier**.  
Ces feuilles reconstituées **peuvent nous apporter des réponses**.

\begin{figure}[H]
\nbdraw{sfce0x0000}
\caption*{Une feuille de Street Fighter II Champion Edition}
\end{figure}

L’analyse de la structure GFXROM de tous les jeux publiés sur CPS-1 révèle toujours **des tuiles regroupées pour correspondre à des dessins réels**,  
ce qui implique clairement l’utilisation du **papier et des ciseaux**.

Mais la disposition des GFXROM commence à changer avec le **premier jeu utilisant le CPS-2**.  
Dans *Super Street Fighter II*, les feuilles des **douze combattants hérités** sont **identiques** à celles utilisées dans *Street Fighter II Champion Edition*.  
En revanche, les nouvelles feuilles de personnages **semblent avoir été créées avec un système automatisé d’allocation**,  
qui **découpe les sprites verticalement**.

L’inspection des GFXROM des jeux suivants indique que **tous les titres sortis sur CPS-2** ont utilisé un **allocateur automatique pour les OBJ**.


 \begin{figure}[H]
\nbdraw{ssf-0x9000}
\caption*{Feuille de Super Street Fighter II (nouveau personnage : Cammy)}
\end{figure}

\img{smc-70_ad.jpg}

\subsection{Numérisation des dessins}\index{Ordinateurs!SONY SMC-70}

Pour numériser leurs dessins, les employés de Capcom utilisaient des ordinateurs **SMC-70**.  
Fabriqué par Sony, le SMC-70 est sorti aux États-Unis et au Japon **à la fin de l’année 1982**.  
Ce qui est particulièrement remarquable avec cette machine, c’est qu’elle a été conçue dès le départ pour être **extensible**.

L’unité principale du SMC-70 regroupe le **clavier** ainsi que les composants centraux : un processeur **Z80 à 4 MHz**, **64 KiB de RAM**, et **64 KiB de VRAM**.  
Le reste du système est **entièrement configurable via des modules d’extension en chaîne**.

La seule limite au système de chaînage réside dans la **capacité de l’alimentation**,  
qui doit impérativement être placée **tout au bout de la chaîne**.

Cette architecture permettait à la machine de Sony d’aller **du simple traitement de texte de bureau** jusqu’à devenir un **puissant outil de traitement vidéo**,  
dans sa configuration la plus poussée.

\begin{figure}[H]
\img{sm70_drawing1.png}
\caption*{Un SMC-70 étendu avec un SMI-7012. L’alimentation se trouve tout à l’arrière}
\end{figure}

\subsubsection{Sony, la grosse machine}

D’un côté, la **longue liste d’extensions et de périphériques** (près de **40 pièces**, toutes fabriquées par Sony) témoigne de l’engagement de la marque dans ce projet.  
De l’autre, cela montre aussi une **volonté claire de contrôler la plateforme**,  
en **empêchant les autres constructeurs** de participer à l’écosystème.


\begin{figure}[H]

\begin{tabularx}{\textwidth}{rrX} 
  \toprule 
  \textbf{Extension Name} & \textbf{Peripheral Name} & \textbf{Function} \\               
  \toprule    
  \texttt{SMI-7011} & & Baie de lecteur disquette 3,5" (interne avec 1 lecteur)\\ 
  \texttt{SMI-7012} & & Baie de lecteur disquette 3,5" (interne avec 2 lecteurs)\\ 
  \texttt{SMI-7013} & & Baie de lecteur disquette 3,5" (externe avec 1 lecteur)\\ 
  \texttt{SMI-7014} & & Baie de lecteur disquette 3,5" (externe avec 2 lecteurs)\\ 
  \texttt{SMI-7016} & & Unité de contrôle des disquettes\\ 
   & \texttt{SMI-7020} & Imprimante matricielle\\ 
  \texttt{SMI-7031} & & Interface série RS232C\\ 
  \texttt{SMI-7032} & & Interface IEEE-488\\ 
  \texttt{SMI-7050} & & Unité de disque cache\\ 
  \texttt{SMI-7056} & & Superchargeur : i8086 à 5 MHz avec 256 KiB de RAM\\ 
   & \texttt{SMI-7060} & Pavé numérique 10 touches\\ 
  \texttt{SMI-7070} & & Convertisseur de signal vidéo\\ 
  \texttt{SMI-7073} & & Superposeur RGB\\ 
  \texttt{SMI-7074} & & Superposeur NTSC\\ 
  \texttt{SMI-7075} & & Vidéotiseur\\ 
  \texttt{SMI-7080} & & Module de sauvegarde par batterie\\ 
  \toprule
  \end{tabularx}%
  \caption*{Extensions et périphériques du SMC-70\cite{smc70tech}}
  \end{figure}
  
  \subsubsection{Capacités notables}
  
  Le SMC-70 est remarquable pour avoir été **le premier ordinateur à proposer un lecteur de disquette 3,5"** (également inventé par Sony en 1981),  
  ainsi que pour sa capacité à **afficher des caractères kanji** grâce à une extension ROM.
  
  Mais c’est surtout par ses **capacités graphiques** que cette machine se démarque.  
  Quatre résolutions étaient disponibles, allant de la **basse résolution 320$\times$200 avec 16 couleurs**,  
  jusqu’à une **haute résolution 640$\times$200 en deux couleurs**.
  
  Le mode **16 couleurs** intéressait particulièrement les artistes de Capcom, car il correspondait **parfaitement au système de palette (pen) du CPS-1**.
  
  \subsection{Tiny Character Editor}
  
  Le SMC-70 ne permettait pas l’utilisation d’un scanner.  
  Le processus de numérisation était **entièrement manuel**.  
  Pour les aider dans cette tâche, les artistes utilisaient un outil appelé **TCE (Tiny Character Editor)**.  
  Aucune capture d’écran n’a jamais été retrouvée, mais les employés de Capcom en ont donné une description sommaire, évoquant son approche minimaliste.
  
  \begin{q}{Koichi Yotsui (planneur de Strider)}
  Tu avais une grille de 16 pixels, une palette de 16 couleurs, et c’est tout.
  \end{q}
  
  \subsection{Dotting}
  
  Pour les éléments **SCROLL** comme **OBJ**, le dotting était réalisé **tuile par tuile**.  
  L’artiste devait regarder son dessin détaillé sur papier quadrillé et décider, pour chaque élément rectangulaire de la tuile,  
  quelle couleur de la palette utiliser.
  
  Le pixel art était un **travail fastidieux et répétitif**,  
  nécessitant un **sens artistique affirmé**, surtout dans les cas où une ligne traversait un pixel.  
  Devait-on laisser tomber ce pixel ? L’afficher en plein ? Ou tenter un **anti-aliasing** en utilisant une couleur voisine de la palette ?  
  Ces choix étaient **délicats**.
  
  Il **n’y avait pas de souris**.  
  Les seules options disponibles étaient **le clavier** ou **le pavé numérique**.  
  Certains employés, insatisfaits de ces méthodes, sont même allés jusqu’à **construire leur propre joystick personnalisé**.
  

\begin{figure}[H]
\img{smc70_keypad.jpg}
\caption*{Un SMC-70 avec pavé numérique. Configuration typique pour le dotting}
\end{figure}

Au minimum, les employés étaient **libres d’utiliser ce qui leur convenait le mieux** en termes d’efficacité.

\begin{q}{Akiman\cite{akiman}}
J’ai utilisé un **clavier** pour dessiner tous les graphismes de *Vampire* et *Street Fighter 2*.
\end{q}

\begin{q}{Akiman\cite{ar20160404}}
  Comme tout était en hexadécimal,  
  on utilisait les **touches de 0 à F** et les **flèches directionnelles** pour fabriquer les sprites.

  Il y avait un gars qui faisait un **boucan monstre en tapant comme un fou sur son clavier**.  
  Il faisait des heures sup’ et ne dormait même pas,  
  alors on n’avait pas le choix : **on devait tous rester éveillés pour continuer à bosser**.
\end{q}

\subsection{Économie de tuiles}

Les artistes tentaient de **réutiliser les tuiles autant que possible**,  
afin de **minimiser l’utilisation de l’espace ROM**, ressource très limitée à l’époque.

Dans *Street Fighter 2*, il n’y avait **assez de GFXROM que pour onze combattants**.  
Ken est ainsi un **assemblage (patchwork)** et une **copie de palette** basé sur les tuiles de Ryu.  
Il “pèse” seulement **98 304 octets**.

Un exploit remarquable si on compare avec des personnages comme :
- **Zangief** (622 592 octets),
- **Honda** (491 520 octets),
- ou **Ryu** (442 368 octets).


\begin{minipage}[t]{0.19\linewidth}
  \sdraw{1.0}{patch_ryu0}
\end{minipage}%
\hfill%
\begin{minipage}[t]{0.19\linewidth}
  \sdraw{1.0}{patch_ryu1}
\end{minipage}
\hfill%
\begin{minipage}[t]{0.19\linewidth}
  \sdraw{1.0}{patch_ryu2}
\end{minipage}%
\hfill%
\begin{minipage}[t]{0.19\linewidth}
  \sdraw{1.0}{patch_ryu3}
\end{minipage}
\hfill%
\begin{minipage}[t]{0.19\linewidth}
  \sdraw{1.0}{patch_ryu4}
\end{minipage}

Créé avec seulement **2 Mio de GFXROM**, *Final Fight* est encore plus impressionnant avec **21 ennemis** et **6 boss**.  
Les sbires sont construits à partir de **sept bases graphiques**, enrichies via des **patchs** et des **palettes**.

\begin{figure}[H]
\nbdraw{ff_template}
\caption*{G. Oriber, Bill Bull et Wong Who de Final Fight}
\end{figure}

\begin{q}{Nin\cite{1991_retro}}
Tout le monde dans l’équipe pensait que *Final Fight* allait bénéficier d’une grande capacité mémoire…  
Mais on s’est trompés.

C’est pour ça que le boss final **Belger saute partout comme ça** :  
on **n’avait pas assez de mémoire** pour ajouter une animation de marche.

Cela dit, **créer quelque chose de stylé avec peu de ressources**,  
c’est comme **un puzzle** pour moi, donc j’ai trouvé ça **plutôt fun**.
\end{q}

\begin{figure}[H]
\nbdraw{0x0100}  
\caption*{Feuille Ryu/Ken}
\end{figure}

**Ryu et Ken utilisent les sept premières couleurs identiques dans leur palette**,  
ce qui facilite la création de **patchs** et la **substitution dynamique**.

\nbdraw{palette_ryu}

\nbdraw{palette_ken}

\begin{figure}[H]
\nbdraw{0x4e00}
\caption*{Feuille de Sagat}
\end{figure}

L’animation de rire de **Sagat** est **doublement optimisée**.  
La séquence ne comporte que **deux poses**, dans lesquelles **seul le buste change**,  
tandis que **les jambes restent les mêmes**.

De plus, la **jambe gauche est absente**.  
Elle est **reconstruite à l’exécution**, en **miroir horizontal** de la **jambe droite**,  
trouvée à l’offset \icode{0xB9}.

\begin{trivia}
La capacité des ASIC à **retourner les tuiles horizontalement** a été largement utilisée dans *Street Fighter II*,  
notamment lorsque les personnages se tournaient vers la gauche ou la droite.

Ce n’était pas un souci pour les personnages **symétriques**,  
mais dans le cas de **Sagat**, son **cache-œil changeait de côté** lorsqu’il se retournait…
\end{trivia}

\pagebreak

\subsection{Structure de l’équipe et culture}

L’équipe artistique fonctionnait avec une **hiérarchie stricte**, basée sur **l’ancienneté** et **les compétences**.


\begin{q}{Akiman\cite{akiman2003}}
  Les planners sont au sommet.\\
  Les artistes seniors travaillent sur les sprites.\\
  Les artistes juniors s’occupent des arrière-plans.
  \end{q}
  
  La structure était **plate**, sans managers intermédiaires.  
  Parmi la **vingtaine de personnes** qui travaillaient sur les graphismes de *Street Fighter II*,  
  **toutes rapportaient à une seule personne : Akiman**\cite{sf2_oral_history}.
  
  Aussi hiérarchisées que soient les tâches,  
  les rôles n’étaient pas figés, et les employés pouvaient **grimper rapidement les échelons**.  
  Akiman, par exemple, a été embauché sur *Dyn Side Arms* en 1986 comme **artiste SCROLL**,  
  le rang le plus bas dans l’équipe artistique.  
  Deux ans plus tard, il devenait **planner** sur *Forgotten Worlds*,  
  et enchaînait avec *Final Fight* et *Street Fighter II* dans ce rôle.
  
  \subsubsection{Éthique de travail}
  
  Une culture du travail exigeante était **instaurée dès le sommet**.
  
  \begin{q}{Akiman\cite{akiman2003}}
    On avait des jours de congé…  
    mais **Yoshiki Okamoto** (chef du développement Capcom) se mettait en colère si tu les prenais.  
    Beaucoup se faisaient engueuler : « Hé, pourquoi t’étais pas là dimanche ?! »
  
    Je pense que **personne ne peut battre mon record** du "taux de vie passée chez Capcom".  
    Pendant les phases de développement, je **dormais toujours sous mon bureau**.
  
    J’y avais même installé un **futon complet** !  
    Quand c’était vraiment tendu, **Yoshiki Okamoto redéfinissait les délais toutes les 10 heures**,  
    donc je ne pouvais plus quitter mon poste…  
    c’est comme ça que j’ai pris **l’habitude de dormir sous mon bureau**.
  
    D’ailleurs, même maintenant que je suis freelance,  
    **je dors encore sous mon bureau à la maison**.
  \end{q}
  
  \subsubsection{Chasse aux talents}
  
  **Conserver les talents** était une priorité stratégique.  
  L’écran de crédits de *Street Fighter II* illustre bien la prudence de Capcom :  
  **les artistes n’étaient crédités que par leur surnom**.
  
  \begin{figure}[H]
  \img{sf2-credit.png}
  \caption*{L’écran de crédits de Street Fighter II n’utilise aucun nom réel, par peur du recrutement sauvage}
  \end{figure}
  
  \begin{trivia}
  La plus grande reconnaissance possible était une **spécialité indiquée à côté du surnom**.  
  Certains crédits, comme dans *Dynasty Wars*, affichaient des **artistes OBJ** ou **artistes SCR**.
  \end{trivia}
  
  Malgré ces précautions, Capcom a perdu **de nombreux employés au fil des années**.  
  Parmi eux : **Takashi Nishiyama**, créateur de *Street Fighter 1*,  
  qui est ensuite parti diriger *Fatal Fury: The King Of Fighters* pour le **grand rival SNK**\cite{YoshikiOkamotoTakashiNishiyama}.
  
  \subsection{Inspiration}
  
  Pour *Street Fighter II*, les artistes ont puisé leur inspiration dans de nombreuses sources.
  
  Des mangas comme *Yasunori Katō* ont contribué à la création du **Dictateur**,  
  tandis que **Tao**, de *Harmagedon: Genma Wars*, est à l’origine de **Chun-Li**.
  
  **Boxer**, **Ryu**, **Ken**, **Sagat** et **Zangief** ont été inspirés de **personnalités réelles**, respectivement :
  - **Mike Tyson**
  - **Mas Oyama**
  - **Joe Lewis**
  - **Sagat Petchyindee**
  - **Victor Zangiev**
  
\begin{trivia}
  À l’origine appelé **M. Bison**, le boxeur a été renommé **Balrog** pour la version américaine,  
  par crainte d’un procès de la part du célèbre champion de boxe poids lourd américain.
  \end{trivia}
  
  Pour les arrière-plans, **Hollywood et les cassettes VHS** ont été d’un grand secours.
  
  \begin{q}{Akiman\cite{ffdevinterview}}
  Je me souviens avoir monté un petit film de présentation à partir de plusieurs extraits.  
  *Streets of Fire* et *Hard Times* avec Charles Bronson sont ceux que j’ai utilisés à l’époque.  
  En gros, des films de baston.
  
  J’ai vraiment pris les mots du président à cœur –  
  **« Utilisez des films ! »**, qu’il a dit,  
  alors j’ai compris que ça voulait dire : **« Allez-y franchement, plagiez-les ! »**
  \end{q}
  
  Les employés **n’étaient pas payés pour regarder un film**.  
  Ils étaient **payés pour en regarder trois !**
  
  \begin{q}{NiN\cite{ffdevinterview}}
  On n’avait pas beaucoup de temps, donc on avait installé **trois moniteurs**  
  pour pouvoir regarder plusieurs films **en même temps**.
  
  On a suivi les conseils du président :  
  **« Regardez-les tous et inspirez-vous-en ! »**
  \end{q}
  
  \begin{trivia}
  Par coïncidence, le titre japonais de *Hard Times* est **"The Street Fighter"**.
  \end{trivia}
  
  \section{Formes et sprites}
  
  Ce détour historique était important pour comprendre le système de feuilles.  
  Avec cette base, nous pouvons maintenant examiner les **dernières contraintes des GFXROM liées aux OBJ**.
  
  Dans cette couche, les tuiles peuvent être utilisées soit **directement**, soit **regroupées**,  
  ce qui impose des dispositions différentes dans la GFXROM.
  
  \begin{figure}[H]
  \img{ken_stage_design1.png}
  \caption*{Inspiration pour le stage de Ken}
  \end{figure}
  
  \begin{figure}[H]
  \img{ken_stage_design2.png}
  \caption*{Esquisse du stage de Ken\cite{sf2completefiles}}
  \end{figure}
  
  \begin{figure}[H]
  \img{ken_stage_design3.png}
  \caption*{Stage de Ken tel qu’il apparaît en jeu}
  \end{figure}
  
  \subsection{Sprite}
  
  Un **sprite** est un ensemble de tuiles avec des **bordures rectangulaires**.  
  Comme nous le verrons dans la section dédiée à la programmation du m68k,  
  il peut être rendu en émettant **une seule commande de dessin**,  
  qui spécifie :
  - l’offset dans la feuille,
  - la largeur en tuiles,
  - et la hauteur en tuiles.
  
  \begin{figure}[H]
  \nbdraw{honda_alloc1}
  \caption*{Honda en tant que sprite}
  \end{figure}
  
  Bien qu’il soit pratique de pouvoir tout afficher avec une seule commande  
  (et que ce soit la meilleure méthode pour des tuiles largement opaques),  
  c’est aussi **inefficace**.
  
  Un groupe de tuiles avec **beaucoup de transparence** :
  - gaspille de la place précieuse dans la GFXROM,
  - **et compte malgré tout** dans la limite des **256 tuiles affichables par frame** imposée par les puces **CPS-A/CPS-B**.
  
  Une autre limitation (ou avantage, selon le point de vue) est que :
  - une **seule palette** est spécifiée dans la commande de dessin du sprite,  
  - donc **toutes les tuiles doivent utiliser cette même palette**.
  
  \subsection{Shape}
  
  Une méthode bien plus **efficace** et **flexible** est d’utiliser une **Shape**,  
  où la disposition des tuiles peut être **arbitraire**.
  
  Cela nécessite **plusieurs commandes de dessin** (chaque tuile doit être indiquée séparément),  
  mais elles peuvent être **placées n’importe où** dans la GFXROM.
  
  Les **Shapes** présentent **trois avantages** majeurs :
  - elles **économisent de l’espace** de stockage,
  - elles **réduisent le nombre de tuiles** actives à l’écran au strict nécessaire,
  - et elles permettent **une palette différente pour chaque tuile**.
  
  L’exemple de **Honda** montre qu’il suffit de **41 tuiles** en Shape,  
  contre **60 tuiles** si on avait utilisé un Sprite.
  
\begin{trivia}
  Dans *Street Fighter II*, les artistes **se limitaient à une seule palette par personnage**,  
  car la couche **OBJ** était utilisée pour plusieurs choses :  
  les personnages, **la décoration** et l’**interface utilisateur (GUI)**.
  
  C’était **un choix de conception**, rien d’imposé techniquement.  
  Ils auraient tout aussi bien pu utiliser **jusqu’à 16 palettes par personnage**,  
  puisque les objets étaient dessinés via des **commandes Shape** et non Sprite.
  \end{trivia}
  
  \begin{figure}[H]
  \nbdraw{honda_alloc2}
  \caption*{Honda en tant que Shape}
  \end{figure}
  
  Pour autoriser les **commandes de dessin Sprite**,  
  le système de compilation doit **placer les images** selon leur usage prévu.  
  Toutes les tuiles d’un sprite doivent être **disposées comme dans l’image d’origine**.
  
  Mais **l’apparence des feuilles Capcom peut être trompeuse**.  
  Les tuiles dans les couches OBJ sont souvent disposées de manière **visuellement cohérente**,  
  ce qui peut laisser penser qu’elles sont affichées via des Sprites.  
  
  Or, cette disposition servait surtout **à suivre les allocations manuellement**.  
  La plupart des jeux Capcom rendent leurs OBJ via **des Shapes**, pas des Sprites.  
  Les outils modernes, eux, **n’ont plus ce problème**.
  
  À la page \pageref{honda_sheet} se trouve une feuille de Honda,  
  affichant **deux fois la même illustration**.  
  Elle apparaît d’abord **en Sprite**, à l’offset \icode{0x00},  
  sous forme de rectangle.
  
  Puis elle réapparaît **en Shape**.  
  Grâce à l’allocateur automatique de \icode{ccps},  
  les tuiles sont disposées dans **l’ordre où elles sont lues dans l’asset**.  
  Cela ressemble à **une purée visuelle**,  
  mais **prend beaucoup moins d’espace**.
  
  \begin{figure}[H]
  \nbdraw{honda_sheet}
  \caption*{Une feuille, générée avec \icode{ccps}, contenant Honda en Sprite et en Shape}
  \end{figure}
  \label{honda_sheet}
  
  % À l’inverse, une Shape n’a aucune contrainte de disposition.
  