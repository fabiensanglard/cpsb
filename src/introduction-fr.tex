\chapter{Introduction}

L'origine de Capcom remonte à la création de deux sociétés par **Kenzo Tsujimoto** :  
I.R.M. Corporation en 1979, puis sa filiale **Japan Capsule Computers Co** en 1981.  
Implantées dans la préfecture d'Osaka, ces entreprises fabriquaient et distribuaient des jeux électroniques.

Après une fusion en 1981, l'entité résultante, **Sanbi**, fut renommée **Capcom** en 1983.  
La première borne à médailles\cite{medal}, sortie la même année, était un jeu de baseball intitulé *Little League*.

Le surnom de leurs produits, "\textbf{Cap}sule \textbf{Com}puters", résumait bien la philosophie de l'entreprise.  
Avec l'ambition d'aller au-delà des ordinateurs personnels à la mode de l'époque,  
les "coin-ops" étaient présentées comme **des capsules remplies à ras bord de fun vidéoludique**.  
Leur coque extérieure rigide symbolisait aussi la volonté de **protéger la propriété intellectuelle**,  
et d'éviter les copies illégales — souvent de **piètres imitations**.

En 1984, Capcom fait son entrée dans le monde du jeu vidéo avec son premier titre : *Vulgus*.  
Les salles d'arcade étaient un environnement **extrêmement concurrentiel**,  
où une borne n'avait que quelques secondes pour **attirer l'œil d'un joueur**.  
Ce défi était d'autant plus grand pour une société qui, à l'époque, **ne disposait pas de la meilleure technologie**.

\begin{q}{Noritaka Funamizu (alias "Poo"), planneur chez Capcom\cite{planner}}
J'ai toujours vu Capcom comme quelqu'un qui se bat avec un bâton en bambou.  
Nous n'avions pas les moyens d'égaler le matériel de Sega ou de Namco.

Pendant qu'eux couraient en Formule 1, nous, on roulait en Honda.
\end{q}

\begin{figure}[H]
\img{1943.png}
\caption*{1943: The Battle of Midway par Capcom (1987)}
\end{figure}

La métaphore prend tout son sens quand on compare deux titres de 1987 côte à côte :  
**"1943" de Capcom** (réalisé par Poo) face à **"Afterburner" de SEGA**.

La plateforme de SEGA, nommée **X Board**, jouait dans une toute autre catégorie.  
Elle fonctionnait avec **deux processeurs Motorola 68000 cadencés à 12,5 MHz**.  
Le processeur graphique — le **Sega Super Scaler**, cadencé à 50 MHz —  
pouvait gérer à la fois le **redimensionnement** et la **rotation** de **256 sprites**,  
superposés à **deux couches de fond** et une **couche route**.

% Son système audio était animé par un Z80 à 4 MHz, couplé à une puce SegaPCM 16 canaux stéréo, capable de restituer des sons numérisés largement supérieurs à ceux obtenus par synthèse FM.

Son système sonore était piloté par un **Zilog Z80 à 4 MHz**,  
connecté à une puce **SegaPCM stéréo 16 canaux**.  
Il permettait de restituer **de la musique et des sons numérisés**,  
bien plus riches que ceux issus des puces de **synthèse FM** les plus répandues de l'époque.

Dans l'autre coin, la carte de Capcom faisait preuve de **beaucoup de courage** :  
elle embarquait un **processeur Z80 à 6 MHz**,  
et un système graphique capable d'animer **32 sprites**,  
superposés à une **couche de texte** et **deux couches de fond**\cite{1942-tech_specs}.

Le son était géré par un **second Z80 à 3 MHz**,  
produisant **musique et effets sonores via synthèse FM Yamaha**.

\vfill

\begin{figure}[H]
\img{afterburner.png}
\caption*{Afterburner par SEGA (1987)}
\end{figure}
\pagebreak

Malgré ses graphismes modestes, *1943* s'est **honorablement vendu**.  
Il est même devenu **le second jeu arcade à table le plus rentable de 1987**,  
grâce à son gameplay engageant.

Les débuts de Capcom ont vu naître d'autres titres à succès,  
en dépit de leur technologie limitée.  
En 1985, *Wolf of the Battlefield* (alias *Commando*)\index{Jeux!Commando}  
a conquis le monde — et surtout le Royaume-Uni —  
où les tests sur site ont généré **plus de 1 000 commandes**\cite{cgm4}.

\begin{figure}[H]
\img{commando.png}
\caption*{Commando (1985)}
\end{figure}



\vfill

\begin{figure}[!b]
    \img{commando_cabinet.png}
    \caption*{Le distributeur américain, Data East, mettait en avant la rentabilité !}
    \end{figure}
    \pagebreak
    
    Sorti en 1985, *Ghost 'n Goblins*\index{Jeux ! Ghost 'n Goblins} est un autre jeu vidéo emblématique de la capacité de Capcom à faire beaucoup avec peu à ses débuts.
    
    Utilisant la même technologie "bâton de bambou" que *Commando*, et dirigé par le même planificateur (Tokuro Fujiwara) à la tête d'une équipe composée du programmeur Toshio Arima, de l'artiste Masayoshi Kurokawa et de la compositrice Ayako Mori, *Ghost 'n Goblins* fut un autre succès.
    
    Dans ce titre inspiré d'un univers médiéval fantastique, le héros doit secourir sa bien-aimée enlevée. Arthur affronte des hordes de zombies, magiciens, squelettes, Red Arremers, chevaliers volants, et le système de contrôle le plus impitoyable jamais conçu.
    
    L'histoire est bien construite. Les boss — licornes, dragons, Satan et le chef Astaroth — sont bien animés. Malgré sa difficulté cauchemardesque, le jeu a plu aux joueurs.
    
    \vfill
    
    \begin{figure}[H]
    \img{gb.png}
    \caption*{Ghost 'n Goblins (1984)}
    \end{figure}
    \pagebreak
    
    Il est devenu le 10\textsuperscript{e} jeu d'arcade le plus rentable au Japon, et le 9\textsuperscript{e} aux États-Unis.
    \vfill
    \begin{figure}[H]
    \img{gb_flyer.png}
    \caption*{Flyer promotionnel de Ghost 'n Goblins (1984)}
    \end{figure}
    \pagebreak
    
    Ce que Capcom n'avait pas en puissance brute, elle le compensait par l'imagination et le bidouillage. Mais la débrouillardise ne suffisait pas toujours. En 1987, Capcom publie *Street Fighter*, un titre audacieux et innovant, mais un échec commercial.
    
    Dans *Street Fighter*, les joueurs contrôlaient leurs personnages avec un joystick classique, mais utilisaient deux gros boutons pneumatiques pour frapper et donner des coups de pied. Des tuyaux conduisaient l'air jusqu'à la carte-mère, où la pression était mesurée : plus le joueur tapait fort, plus les dégâts infligés étaient importants.
    
    Mais en l'absence de retour tactile, les joueurs avaient tendance à frapper aussi fort que possible, sans gérer leur effort. Après quelques rounds, le bras droit était épuisé, et le jeu cessait d'être amusant à cause de la fatigue. En plus du manque de plaisir, il y avait des risques de blessures et de tendinites. Pour y remédier, Capcom équipa plus tard les bornes d'un système de contrôle plus "standard", avec six boutons classiques permettant tout de même de choisir la puissance du coup.
    
    Le jeu devint jouable, mais resta plombé par des graphismes peu impressionnants et une jouabilité poussive. La borne fut largement ignorée, et malgré des tentatives de relance via des réductions, "la borne dédiée la plus incroyable jamais conçue" sombra rapidement dans l'oubli.
    
    \begin{figure}[H]
    \img{sf1_4_3.png}
    \caption*{Street Fighter 1 (1987)}
    \end{figure}
    
    \begin{figure}[H]
    \img{sf1_cabinet.png}
    \caption*{Flyer de la borne Street Fighter 1}
    \end{figure}
    
    \section{Une production coûteuse}
    En plus de faire face à des concurrents technologiquement supérieurs, Capcom devait composer avec une chaîne de production en constante évolution. L'étude des \textbf{cartes électroniques imprimées} (\textbf{PCB})\index{PCB!Cartes électroniques imprimées}\index{PCB} utilisées pour héberger leurs jeux entre 1984 et 1988 révèle une grande diversité de composants.
    
    Un simple tableau résumant l'usage par Capcom des processeurs Motorola 6809, Zilog z80, Motorola m68k, Intel 8751 (MCU), ainsi que des puces sonores YM2203, YM2151, YM2149, et MSM5205 montre que même des jeux sortis la même année pouvaient embarquer des puces totalement différentes.
    
    Même le très courant z80 était utilisé de façon non uniforme, pouvant être dédié à la logique de jeu, au son, ou aux deux — comme dans la carte de "1942", qui en contient deux exemplaires.
    



\begin{figure}[H]
{ 
\setlength{\tabcolsep}{3.0pt}
\setlength\cmidrulewidth{\heavyrulewidth} % Make cmidrule = 
\begin{tabularx}{\textwidth}{Xccccccccc}

  & & \multicolumn{4}{c}{CPU} &  \multicolumn{4}{c}{SOUNDS} \\
  \cmidrule(lr){3-6}
  \cmidrule(lr){7-10}
 
  \textbf{Game Name} & \textbf{Year} & \textbf{M6809} & \textbf{z80} & \textbf{m68k} & \textbf{i8751} & \textbf{2203} & \textbf{2151} & \textbf{2149} & \textbf{5205} \\               
  \toprule    
\href{https://www.youtube.com/watch?v=45ELzG1ivEA}{Vulgus}
                & 1984          &               &      X       &              &              &               &               &       X       &               \\
\href{https://www.youtube.com/watch?v=R5mg6XPqtBs}{Higemaru}
                & 1984          &               &      X       &              &              &               &               &       X       &               \\
\href{https://www.youtube.com/watch?v=Em7UwOOBvlA}{1942}
                & 1984          &               &      X       &              &              &               &               &       X       &               \\
  \toprule    
\href{https://www.youtube.com/watch?v=1qctKI_t5eY}{Commando}
                & 1985          &               &      X       &              &              &       X       &               &       X       &               \\
\href{https://www.youtube.com/watch?v=SugLAqaPhqA}{Ghost 'n Goblins}  
                & 1985          &       X       &      X       &              &              &       X       &               &       X       &               \\
\href{https://www.youtube.com/watch?v=mrO9qwGXdy8}{Gun Smoke}        
                & 1985          &               &      X       &              &              &       X       &               &       X       &               \\
\href{https://www.youtube.com/watch?v=cIC2mNNryZg}{Section Z}
                & 1985          &               &      X       &              &              &       X       &               &       X       &               \\
  \toprule    
\href{https://www.youtube.com/watch?v=L1FVWdlQNG8}{Trojan}
                & 1986          &               &      X       &              &              &       X       &               &               &       X       \\
\href{https://www.youtube.com/watch?v=57lg9pFUgco}{Speed Rumbler}
                & 1986          &       X       &      X       &              &              &       X       &               &       X       &               \\
\href{https://www.youtube.com/watch?v=0QyLx94PMio}{Dyn Side Arms}
                & 1986          &               &      X       &              &              &       X       &               &       X       &               \\
\href{https://www.youtube.com/watch?v=0f4jWQyf-fs}{Legendary Wings}
                & 1986          &               &      X       &              &              &       X       &               &       X       &               \\
  \toprule    
\href{https://www.youtube.com/watch?v=kntCwchJWfw}{1943}
                & 1987          &               &      X       &              &              &       X       &               &       X       &               \\
\href{https://www.youtube.com/watch?v=ZzKStmMAiHM}{Black Tiger}
                & 1987          &               &      X       &              &      X       &       X       &               &       X       &               \\
\href{https://www.youtube.com/watch?v=kVLCv-YgWco}{Street Fighter}
                & 1987          &               &      X       &      X       &      X       &               &       X       &               &       X       \\
\href{https://www.youtube.com/watch?v=1ZtwOGN-ZeE}{Tiger Road}
                & 1987          &               &      X       &      X       &              &       X       &               &       X       &       X       \\
  \toprule    
\href{https://www.youtube.com/watch?v=zG620nr7vko}{Bionic Commando}
                & 1988          &               &      X       &      X       &      X       &               &       X       &               &               \\
\href{https://www.youtube.com/watch?v=zG620nr7vko}{F1-Dream}
                & 1988          &               &      X       &      X       &      X       &       X       &               &       X       &               \\
  \toprule    
\end{tabularx}%
}\caption*{Utilisation des puces dans les bornes d'arcade Capcom entre 1984 et 1988\cite{cps0chipslist}.}
\label{fig:capcom_pcbs}
\end{figure}

Même si les jeux faisaient globalement la même chose (déplacer des sprites sur des arrière-plans), le matériel devait constamment être réinventé.

Cette évolution perpétuelle ralentissait la production, car le rythme de développement était freiné par les bugs matériels. La pleine vitesse de programmation n'était atteinte qu'en fin de projet, ce qui plaçait Capcom dans une position désavantageuse supplémentaire.

\section{Miné par le piratage}
Comme les cartes électroniques de Capcom étaient fabriquées à partir de composants standards du marché, les contrefacteurs pouvaient les copier, extraire les ROMs des logiciels et fabriquer des répliques appelées *bootlegs*.

Sans avoir à amortir les coûts de développement, ces copies étaient vendues à un prix inférieur à celui des jeux officiels. Les ventes manquées pesaient lourdement sur la santé financière de Capcom.

\section{Capcom NT (Nouvelle Technologie)}
Les difficultés de production, la concurrence et le piratage peignaient un avenir incertain pour la division arcade de Capcom. Mais l'histoire a montré qu'ils ne se sont pas seulement maintenus… ils ont prospéré.

Une nouvelle ère commence en 1988.  
*Forgotten Worlds* et *Strider* offrent aux joueurs un premier aperçu de ce que la société d'Osaka est désormais capable de produire.

\label{nin_fw}
\begin{figure}[H]
\img{fw_4_3.png}
\caption*{Forgotten Worlds (1988)}
\end{figure}

\label{fw_flyer}
\begin{figure}[H]
\img{cps1_announcement.jpg}
\caption*{Flyer d'annonce du CPS-1 par Capcom (1989)}
\end{figure}

Ces deux jeux tournaient sur une mystérieuse "nouvelle technologie", plus tard renommée \textbf{CP-System} ou \textbf{CPS-1}, ce qui permit une nette amélioration de la qualité de production.  
D'énormes sprites se déplaçaient à l'écran. Ils étaient composés d'un plus grand nombre de couleurs, répartis sur plusieurs couches simulant un effet de parallaxe. Les niveaux étaient plus élaborés (*Strider* comportait même des surfaces en pente escaladables). Côté son, les améliorations incluaient des bruitages numérisés et des musiques à base d'échantillons.

Le premier méga-succès de Capcom arrive avec *Final Fight*\index{Jeux ! Final Fight} en 1989. Jusqu'à cette époque, le genre *beat ‘em up* était dominé par Technos grâce à son excellente série *Kunio-kun* (connue hors du Japon sous le nom *Renegade*) et surtout le carton *Double Dragon*.

Avec Cody, Guy et Mike, Capcom frappe fort.  
Malgré un budget minuscule de seulement 2 Mio pour les graphismes, l'équipe artistique menée par Akira Yasuda (alias Akiman) exploita tout le potentiel du CPS-1 pour offrir des visuels somptueux et une ambiance sonore accrocheuse. Le gameplay était excellent, avec de nombreux ennemis, des boss et des héros aux compétences variées.  
Mais surtout, le timing était parfait pour le marché américain, où le genre "beat‘em up" faisait fureur.




  

\label{nin_ff}
 \begin{figure}[H]
\img{ff_4_3.png}
\caption*{Final Fight (1989)}
\end{figure}

"Final Fight" devint rapidement le jeu le plus vendu de Capcom\cite{birth_of_chunli}, établissant définitivement l'entreprise comme une référence incontournable dans le monde de l'arcade.

\pagebreak

Le plus beau compliment est peut-être venu de ses concurrents eux-mêmes, qui, des années plus tard, avouèrent combien *Final Fight* les avait démoralisés.

\begin{q}{Yoshihisa Kishimoto, game planner (Double Dragon \& Kunio-kun)\cite{dd} }
Les gens de Capcom ont porté un coup très dur à Technos Japan avec *Final Fight*, qui surpassait *Double Dragon III* à tous les niveaux. Non seulement ils avaient des designers incroyables, mais en plus ils donnaient à leurs équipes les moyens d'innover grâce au CPS-1.

Pour nous, ce fut un réveil brutal : cela prouvait que nous n'avions pas su évoluer aussi vite qu'eux.
\end{q}

Et ce n'est pas seulement la qualité qui s'est améliorée, mais aussi la quantité.  
Grâce à une plateforme stable et à des outils bien conçus, Capcom put sortir plus de trente jeux entre 1988 et 1995, tous développés sur sa plateforme CPS-1.

Parmi ces titres figurait la suite de *Street Fighter*, qui allait conquérir le monde.

\label{nin_sf2}
\begin{figure}[H]
\img{sf2_4_3.png}
\caption*{Street Fighter 2 (1991)}
\end{figure}

\index{Games!All from 1988 to 1995}
\begin{figure}[H]
{ 
\setlength{\tabcolsep}{3.0pt}
\begin{tabularx}{\textwidth}{Xrrrrrrr}
  \textbf{Game Name} & \multicolumn{5}{c}{ \textbf{Type} } &\textbf{ GFX }  & \textbf{ Year } \\                 
  \toprule    
\href{}{Forgotten Worlds} & &  & \shmupcube & & & 4 MiB & 1988 \\ 
\href{}{Ghouls'n Ghosts} & & \platcube & & & & 3 MiB & 1988 \\ 
  \toprule    
\href{}{Strider} & & \platcube & & & & 4 MiB & 1989 \\ 
\href{}{Dynasty Wars} & & &  & \beatallcube & & 8 MiB & 1989 \\ 
\href{}{Willow} & & \platcube & & & & 4 MiB & 1989 \\ 
\href{}{U.N Squadron} & &  & \shmupcube & & & 2 MiB & 1989 \\ 
\href{}{Final Fight} & & & & \beatallcube & & 2 MiB & 1989 \\ 
  \toprule    
\href{}{1941: Counter Attack} & &  & \shmupcube & & & 2 MiB &  1990 \\ 
\href{}{Mercs} & \ocube & & & & &  3 MiB & 1990 \\ 
\href{}{Mega Twins} & & \platcube & & & & 2 MiB & 1990 \\ 
\href{}{Magic Sword} & & \platcube & & & & 2 MiB & 1990 \\ 
\href{}{Carrier Air Wing} & &  & \shmupcube &  & & 2 MiB  & 1990 \\ 
\href{}{Nemo} & & \platcube & & & & 2 MiB &  1990 \\ 
  \toprule    
\href{}{Street Fighter II: The World Warrior} & & & & & \duelcube & 6 MiB & 1991 \\ 
\href{}{Three Wonders} & \ocube & & & & & 4 MiB & 1991 \\ 
\href{}{The King of Dragons} & & & & \beatallcube& & 4 MiB & 1991 \\ 
\href{}{Captain Commando} & & & & \beatallcube& &  4 MiB & 1991 \\ 
\href{}{Knights of the Round} & & &  & \beatallcube& & 4 MiB  & 1991 \\ 
  \toprule    
\href{}{Street Fighter II: Champion Edition} & & & & & \duelcube & 6 MiB & 1992 \\ 
\href{}{Adventure Quiz: Capcom World 2} & \ocube& & & & & 2 MiB & 1992 \\ 
\href{}{Varth: Operation Thunderstorm} & & & \shmupcube &  & & 2 MiB & 1992 \\ 
\href{}{Quiz \& Dragons: Capcom Quiz Game} & \ocube & & & & & 2 MiB & 1992 \\ 
\href{}{Street Fighter II' Turbo: Hyper Fighting} & & & & & \duelcube &  6 MiB & 1992 \\ 
  \toprule    
\href{}{Ken Sei Mogura: Street Fighter II} & \ocube & & & & & 6 MiB & 1993 \\ 
\href{}{Pnickies} & \ocube & & & & &  2 MiB & 1993 \\ 
  \toprule    
\href{}{Quiz Tonosama no Yabo 2} & \ocube & & & & &  4 MiB & 1995 \\ 
\href{}{Pang! 3} & & \platcube & & & & 2 MiB  & 1995 \\ 
Mega Man the Power Battle & & & & & \duelcube &  8 MiB  & 1995 \\

\toprule    
\end{tabularx}%
}\caption*{CPS-1 games: \ocube{} Other, \platcube{} Platform, \shmupcube{} Shmup, \beatallcube{} Brawl, \duelcube{} Duel}
\end{figure}



Situé à la croisée des chemins entre la nouvelle technologie de Capcom et les compétences de "faire plus avec moins" que ses équipes avaient acquises par nécessité, *Street Fighter 2*\index{Jeux ! Street Fighter 2} fut un bond quantique dans l'histoire du jeu vidéo — un véritable phénomène.

Les huit personnages avaient des tailles, des morphologies et des genres différents. Chacun disposait de ses propres mouvements et capacités spéciales. Cette caractérisation leur donnait de la profondeur. La musique était entraînante, les effets sonores percutants. La précision des contrôles incitait les joueurs à progresser, à affiner leurs compétences et à maîtriser leur avatar.

Le matériel permettait un effet de parallaxe ligne par ligne à couper le souffle, affiché à un taux constant et fluide de 60 Hz, mettant en valeur le talent de l'équipe artistique.

\begin{figure}[H]
\img{sf2_tournament.png}
\caption*{Flyer d'un tournoi Street Fighter 2}
\end{figure}

Le jeu développa instantanément une communauté de fans fidèles. Il fallait faire la queue pour insérer sa pièce. Acheter un "Continue" était mal vu par les autres joueurs impatients d'avoir leur tour.  
Les exploitants achetaient plusieurs bornes du jeu pour réduire les temps d'attente… et en fin de journée, les machines débordaient tout de même de pièces\cite{sf2_oral_history}.

La popularité était telle que des tournois, avec des récompenses alléchantes (voir page opposée), furent organisés.

En 1995, la série avait généré 2,3 milliards de dollars de revenus avec 200 000 bornes vendues\cite{usgamer20160101} (60 000 unités *World Warrior* et 140 000 *Champion Edition*).  
En 2017, ce chiffre atteignait 10,61 milliards de dollars\cite{gamerevolution20140126}, faisant de *Street Fighter 2* le troisième jeu vidéo le plus rentable de tous les temps.

\begin{trivia}
As-tu remarqué ce joueur omniprésent surnommé \textbf{NiN}, qui détient tous les meilleurs scores de *Forgotten Worlds* (p\pageref{nin_fw}), *Final Fight* (p\pageref{nin_ff}) et *Street Fighter II* (p\pageref{nin_sf2}) ?  
C'est le pseudonyme d'Akira Nishitani, le game planner de tous ces jeux !
\end{trivia}

\section{Ode au CP-System}

Ce livre est une déclaration d'amour d'ingénieur au système qui a permis à Capcom de passer d'une entreprise en lutte pour sa survie à un nom incontournable de l'arcade.

L'objectif de ce travail est de comprendre le CP-System, depuis les fondations. Cela sera (espérons-le) atteint en exposant d'abord le matériel, puis en montant progressivement jusqu'aux niveaux de programmation et d'architecture des moteurs de jeu.

\subsubsection{Matériel}
Le CP-System est composé de quatre sous-systèmes matériels détaillés dans le premier chapitre :
\begin{itemize}[topsep=0pt]
\item Système de contrôle
\item Système audio
\item Système graphique (GFX)
\item Système vidéo
\end{itemize}

Au-delà de la réalité physique des circuits et des bus, une réflexion est menée sur les choix de conception, ainsi que des exemples concrets montrant comment les jeux ont tiré parti de ces fonctionnalités.

\subsubsection{Logiciel}
Les chapitres suivants étudient le logiciel et la manière de le construire. En particulier, les quatre groupes de ROM constituant un jeu sont expliqués :

\begin{itemize}[topsep=0pt]
\item ROM du Motorola m68k
\item ROM de programmation du Zilog z80 et ROM musicale du YM2151
\item ROM audio du MSM6295 (échantillons sonores)
\item ROM graphique (GFX) du CPS-A/CPS-B
\end{itemize}

Ces chapitres utilisent des outils modernes, mais comportent également une section "À l'époque", explorant comment Capcom travaillait dans les années 90.

\subsubsection{Humain (Peopleware)}
Les personnes impliquées dans le matériel ou le logiciel sont citées dans les sections correspondantes.  
Cependant, Capcom étant déjà une grande entreprise au début des années 90, de nombreuses personnes ont participé à l'histoire du CPS-1.  
Pour aider le lecteur à suivre tous les acteurs, une liste des contributeurs et de leurs rôles est disponible à la page \pageref{people}.

