\chapter*{Remerciements}

Merci à \textbf{Victoria Ho} pour la relecture de ce livre et pour avoir évité un carnage grammatical.

Merci à \textbf{Lo\"{i}c Petit} d'avoir généreusement partagé son expertise sur le CPS-1 et le CPS-2. La section consacrée aux ASICs graphiques CPS-A et CPS-B n'aurait pas été possible sans la documentation détaillée qu'il a produite. Il s'est également porté volontaire pour relire de nombreuses versions préliminaires de ce livre et a suggéré de nombreuses améliorations importantes. Sa contribution a été inestimable.

L'autre géant sur les épaules duquel ce livre repose est \textbf{Upsilandre}. Il a consacré beaucoup de temps à m'expliquer le ratio d'aspect des pixels et a repéré de nombreuses erreurs lors des relectures. Il a mené et partagé de nombreuses recherches, non seulement sur le CPS-1 mais aussi sur le X68000. Il a également partagé des articles et une méthodologie pour explorer les portages Capcom sur \href{https://www.gamopat-forum.com}{gamopat-forum.com}, particulièrement éclairants.

Merci à \textbf{Charles MacDonald} pour avoir partagé ses connaissances sur le CPS-1 et pour avoir patiemment expliqué — et réexpliqué — l'art des circuits de tilemaps.

Merci à \textbf{Ben Torkington} d'avoir mis à disposition "SF2:Platinum", sa réécriture en ANSI C de "Street Fighter II: World Warriors".

Merci à \textbf{STG} pour sa traduction d'articles japonais sur \href{https://shmuplations.com}{shmuplations.com}.

Merci à \textbf{VGDensetsu} pour ses nombreux articles sur la série Street Fighter et le développement de jeux japonais en général.

Merci à \textbf{mvs-scans.com} pour leurs photos de haute qualité des cartes CPS-1.

Merci à \textbf{John McMaster} pour ses scans haute résolution de l'ASIC CPS-A.

Merci aux \textbf{contributeurs de M.A.M.E} pour leur travail de documentation colossal réalisé au fil des années. Vous êtes les héros méconnus de l'histoire de l'arcade.

Merci à \textbf{Mike Stedman} pour avoir mis en lumière certaines des fonctionnalités et des portages les plus obscurs du X68000.

Merci à \textbf{James Young (Pronoiac)} pour ses nombreuses pull requests visant à améliorer la grammaire de ce livre.

Merci à celles et ceux qui ont pris le temps de signaler des bugs et des coquilles : iridium87.
