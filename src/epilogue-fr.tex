\chapter{Épilogue}

L'étude du CPS-1 fut un projet de passion qui m'a occupé pendant plus d'un an sur mon temps libre. Mon objectif était d'explorer obsessionnellement le matériel, de le comprendre jusqu'au plus bas niveau, et d'apprendre à le programmer. Comme c'est souvent le cas, le voyage a pris une tournure inattendue, et je suis ressorti de cette aventure avec bien plus que ce que j'avais imaginé.

Au départ, découvrir les entrailles de la machine de Capcom était fascinant, voire presque addictif. Il m'arrivait souvent de passer des nuits blanches à explorer des schémas électroniques ou à expérimenter du code. La technologie que j'ai dévoilée a confirmé le rôle crucial qu'elle a joué dans la trajectoire de Capcom.

C'est lorsque j'ai commencé à étudier les systèmes concurrents du CP-System que mon point de vue a commencé à évoluer.

Le rival juré de Capcom, SNK, avait construit une machine impressionnante, techniquement supérieure au CP-System. Les jeux étaient conçus uniquement à base de sprites, sans utiliser les tilemaps contraignants. Tandis que le CPS-1 pouvait afficher jusqu'à 256 sprites, la Neo-Geo en gérait 381. Chacun des sprites de la Neo-Geo pouvait être redimensionné à l'aide d'une technique de rétrécissement, largement utilisée dans des titres à succès comme *Super Sidekicks*.

Et la liste des fonctionnalités continue. Les programmeurs pouvaient définir des animations que le matériel gérait automatiquement — une fonctionnalité exploitée à profusion dans *Metal Slug* pour obtenir ces visuels somptueux qui ont fait sa renommée. La détection HSYNC permettait des effets de trame. La capacité de 330 mégabits de ses cartouches était fièrement affichée.

Et pourtant, malgré les limites techniques de son matériel, Capcom réussissait à faire jeu égal. À plusieurs reprises, ses jeux ont même surpassé en succès ceux tournant sur Neo-Geo. C'était comme si, passé un certain seuil, la technologie ne comptait plus vraiment.

Alors que ce livre touche à sa fin, je me surprends à admirer de plus en plus le travail de celles et ceux qui ont insufflé la vie dans ce silicium. Oui, ils disposaient d'une bonne plateforme, mais ce n'était pas une solution miracle. Ces créatifs dormaient sous leur bureau. Ils traquaient courageusement les allocations mémoire avec du papier et des ciseaux, saisissaient les couleurs des pixels à la main, tuile par tuile, à l'aide d'un clavier. Ils travaillaient tard dans la nuit et faisaient passer des puces ROM par la fenêtre attachées à une ficelle pour respecter les délais.

Ce projet avait pour but de faire apprécier davantage le matériel aux lecteurs. Il se termine par un auteur qui a ouvert les yeux sur les artistes et designers qui ont mis une âme dans la machine.

- Fabien Sanglard
