\chapter{Epilogue} 

The "CPS-1 study" was passion project that took me over a year to complete and resulted into an epiphany.

Discovering the internals of Capcom's hardware was a fascinating experience. Its technology was an indeniable instrument that not only allowed Capcom to survive, it contributed to making it an icon. 

The CPS-1 did a fantastic job
at fighting piracy and streamline production. Its tile engine was a solid and impressive technology at the time. But if we look at the visual effects and audio it could output, and compare it to what competitors lined up, it was not an outstanding system. The difference is especially striking if you line up next to its main rival's, SNK, flagship. 

In a Neo-Geo a console archeologist will find sprite-shrinking (extensively used in Super Sidekicks),
auto-animation (a feature allowing to define all frames in an animation and to fire-forget found in profusion in Metal Slug), and
the capability to detect HSYNC to implement gorgeous raster effects. And what to say about the 700+ megabits capacity?

The short answer is that it did not matter.

Technology is a vector which when is well executed will take a company 50\% toward the finish line. But it is not all of it.

Taking a cabinets to another level is where the artist and designers must shine. And shine they did. 

If I started this project with an obsession toward the harware, resolved to understand it to the deep down, I emerged with a tremendous respect for the all these people who word tirelessly to put a soul in the machine. 

Now, can please somebody write the Neo-Geo book so I don't have to?
