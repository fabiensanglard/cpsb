\chapter{Epilogue} 

The CPS-1 study was a passion project that took me over a year to complete. As a technology inclined person, the goal was to obsessively understand the machine to the deep down.

Discovering the internals of Capcom's hardware was fascinating and borderline addictive. I often found myself in the wee-hours exploring schematics or experimenting with code. The technology that unraveled under my eyes made it undeniable that this machine played a key part in crafting Capcom's success. 

The CPS-1 did a fantastic job at fighting piracy and streamline production. But if its tile engine was a solid and flexible solution to developers' problems, the comparison with other system puzzled me.

Having studied the evolution of 3D game engines in the 90s I was biased towards the importance of graphic superiority. How could I justify the success of the CPS-1 against the Neo-Geo? 

On paper, SNK's machine could do more than Capcom's. It could scale via sprite-shrinking (extensively used in Super Sidekicks),
had auto-animation (a feature allowing to define all frames in an animation and to fire-forget found in profusion in Metal Slug), and the capability to detect HSYNC to implement gorgeous raster effects. And what to say about the 700+ megabits capacity?

The answer to my question is that, past a certain point technology cease not matter.

Technology is a vector which when is well executed will take one 50\% toward the finish line. But it is not all of it. Taking a product to another level is where artists and designers must shine. Those are the people that put a soul in the machine.

- Fabien Sanglard

PS: Now, can somebody, please, write the Neo-Geo book so I don't have to?
