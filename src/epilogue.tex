\chapter{Epilogue} 

The CPS-1 study was a passion project that took over a year to complete in my spare time. The goal was to obsessively explore the hardware, understand it down to the metal, and learn how to program it. As it is often the case, the journey took an unexpected turn and I came out with  more than I initially anticipated.

In the beginning, discovering the internals of Capcom's machine was fascinating and borderline addictive. I often found myself in the wee-hours exploring schematics or experimenting with code. The technology that unraveled confirmed the key part it played in shaping Capcom's destiny. 

The CP-System was an arcade landmark. It did a fantastic job at fighting piracy and streamline production. Its tile engine was a solid and flexible solution to developers' problems which let artists express their creativity. The audio chips allowed musicians to produce melodies rivaling with competitions.

It is when I studied sideway to learn about the technology of Capcom competitors that it stuck me. 

The Neo-Geo, the CP-System main competitor at the time, was an impressive machine surpassing Capcom's flagship from a technology standing point. SNK's machine was completely designed around sprites, removing the limitation of SCROLLs. Additionally, each sprite could be scaled via a shrinking technique which was extensively used in successful titles such as Super Sidekicks. The list goes one with auto-animation (a feature allowing to define all frames in an animation and to fire-forget used profusely in Metal Slug), HSYNC detection to implement gorgeous raster effects, and the famous title worthy 700+ megabits capacity of its boards.

Despite the hardware's shortcomings and its non-square pixels, the games running on CP-System held their own. Some of them even managed to become world-wide phenomenon. For these titles, it was almost like if technology did not matter.

As a programmer, I often focus on performance, speed, and efficiency. 

As my exploration neared its end, I found myself admiring more and more the work of the people who breed life into the silicon.   
They courageously tracked allocations with paper and scissors, they entered pixel colors by hand on a keyboard, they worked over night, exchanging the result of their work via the windows. Their creativity, devotion, and pragmatism were the other 50\% needed to ship great games.

The CP-System was a marvelous machine which brought important technological advantage to the table. But it is also thanks to the devotion of its artists and designers that Capcom ended up with a soul in its machine.

Lucky us.

- Fabien Sanglard
