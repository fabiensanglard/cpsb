\chapter{Epilogue} 

The CPS-1 study was a passion project that took over a year to complete in my spare time. The goal was to obsessively explore the hardware, understand it down to the metal, and learn how to program it. As it is often the case, the journey took an unexpected turn and I came out of the adventurede with more than I initially anticipated.

In the beginning, discovering the internals of Capcom's machine was fascinating and borderline addictive. I often found myself in the wee-hours exploring schematics or experimenting with code. The technology that unraveled confirmed the key part it played in shaping Capcom's destiny. 

The CP-System is an landmark in the history of arcades. It did a fantastic job at fighting piracy and streamline production. Its tile engine was a solid and flexible solution to developers' problems which let artists express their creativity. The audio chips allowed musicians to produce melodies rivaling with competitions.

It is precisely when studying the comtemporary systems technology stack that something stuck me. 

The Neo-Geo, the CP-System main competitor at the time, was an impressive machine which surpassed Capcom's flagship. SNK games were built relying exclusively on sprites without using limitating tilemaps. While the CPS-1 could display 256 sprites, the Neo-Geo could achieve 381. Each of the Neo-Geo sprites could be scaled via a shrinking technique extensively used in successful titles such as Super Sidekicks. 

The list goes on with auto-animation (allowing to define-and-forget an animation, a feature used profusely in Metal Slug for the georgeous result we all know), HSYNC detection which unlocked raster effects, and the proudly advertised 330 megabits capacity of its boards.

Despite the hardware's shortcomings and its non-square pixels, the CP-System had games able to held their own. Several of them even managed to become world-wide phenomenon. For these titles, it was almost like if technology did not matter.

As this book was coming to an end, I found myself admiring more and more the work of the people who breed life into the silicon. Yes, they had a good platform to work with. But they also slept under their desk. They courageously tracked allocations with paper and scissors, they entered pixel colors by hand, tile by tile, using a keyboard. They worked over night and passed ROM chips using string though the windows. 

This trip was started to nurture love for the CP-System hardware. It ended with a profund appreciation for the artists and designers who put a soul in the machine.

- Fabien Sanglard
