\renewcommand{\bibname}{Notes \&\ References}
\begin{thebibliography}{999}

\addcontentsline{toc}{chapter}{Notes \&\ References}

\bibitem{medal} 
\textbf{"Qu'est ce qu'un jeu de médaille ?"}
Ce n'est pas une erreur ! Un jeu à médailles se joue avec des pièces métalliques. Les plus connus sont les "pusher games", dans lesquels le joueur doit déposer des pièces sur un système de plateformes. Chaque plateforme avance et recule comme des balais automatiques. Le but est de faire tomber des groupes de pièces au-delà du bord de la dernière plateforme, pour que le joueur les récupère en récompense.

\bibitem{planner} 
\textbf{"À propos du rôle de Planner"}
Le "Planner" était le principal décideur au sein d'une équipe de développement de jeux au Japon. Chargé de définir les grandes orientations et de prendre les décisions de game design, tous les autres membres de l'équipe lui rendaient compte. Il y avait généralement un seul Planner en charge (comme Poo sur 1943: The Battle of Midway), mais il pouvait y en avoir deux, comme dans Street Fighter II où Akira Nishitani (Nin) et Akira Yasuda (Akiman) occupaient tous deux ce rôle.

\bibitem{1942-tech_specs} 
  \textbf{"Analyse finale de 1942 par la Team Arcade"} (by Tyler Huberty, Greg Nazario, Isaac Simha,
  \href{https://course.ece.cmu.edu/~ece545/F16/reports/F12_Arcade1942.pdf}{link},
  2012-09-12.


\bibitem{cgm4} 
  \textbf{"Computer Gamer Magazine \#4"} ("Coin-Op Connection" article,
  \href{https://www.youtube.com/watch?v=RxIXilYv0kM}{link},
  1985-07.

\bibitem{becareful} 
  \textbf{"Intérrogation sur les chiffres"},
  Les chiffres "deux ans et cinq millions de dollars" doivent être considérés avec prudence. Ces données proviennent d'un flyer promotionnel de Forgotten Worlds (\pageref{fw_flyer}), qui mentionnait également trois processeurs Motorola 68000, alors que le produit final n'en contenait qu'un seul.
  1989.

\bibitem{TheStoryOf3Dfx} 
  \textbf{"L'histoire de la 3Dfx"} (by Fabien Sanglard),
  \href{https://fabiensanglard.net/3dfx_sst1/index.html}{link},
  2019-04-04.

\bibitem{fm_licensing} 
  \textbf{"The Sound of Innovation: Stanford and the Computer Music Revolution"} (by Andrew J. Nelson),
  ISBN: 978-0262028769.
  2015-03-06

\bibitem{birth_of_chunli} 
  \textbf{"La naissance de Chun-Li"} (Akiman for Archipel),
  \href{https://www.youtube.com/watch?v=RxIXilYv0kM}{link},
  2018-02.

\bibitem{mips} 
  \textbf{"Déclarations de performances informatiques de 1980 à 1996"} (Roy Longbottom),
  \href{http://www.roylongbottom.org.uk/mips.htm}{link}.

\bibitem{dd} 
  \textbf{"Les grands noms du jeu video, Yoshihisa Kishimoto - Enter the Double Dragon"} (Florent Gorges for Editions PixNlove),
  \href{https://www.editionspixnlove.com/les-grands-noms-du-jeu-video/289-yoshihisa-kishimoto-enter-the-double-dragon.html}{link},
  2012-07-05.


\bibitem{akiman} 
  \textbf{"Le compte X de Akiman"} (akiman),
  \href{https://twitter.com/akiman7/status/465507673572519936}{post 1},
  \href{https://twitter.com/akiman7/status/309615270815731712}{post 2},
  \href{https://twitter.com/akiman7/status/386598518380453888}{post 3}.

\bibitem{usgamer20160101}
  \textbf{"Top 10 des jeux d'arcade les plus rentables de tous les temps"} (Jaz Rignall for usgamer.net) (200,00 units: SF2 WW sold 60,000 while SF2 CE sold 140,000),
  \href{https://www.usgamer.net/articles/top-10-biggest-grossing-arcade-games-of-all-time}{link},
  2016-01-01.

\bibitem{gamerevolution20140126}
  \textbf{"World of Warcraft en tête de l'industrie avec près de 10 milliards de dollars de revenus"} (Jonathan Leack for gamerevolution.com),
  \href{https://www.gamerevolution.com/features/13510-world-of-warcraft-leads-industry-with-nearly-10-billion-in-revenue#/slide/1}{link},
  2014-26-01.
  
\bibitem{retro}
  \textbf{"Interview de Noritaka Funamitsu"} (Retro magazine),
  \href{http://fightingstreet.com/folders/variousinfofolder/interviewfolder/sfii_funamitsu/funamitsu1.jpg}{part 1},
  \href{http://fightingstreet.com/folders/variousinfofolder/interviewfolder/sfii_funamitsu/funamitsu2.jpg}{part 2},
  \href{http://fightingstreet.com/folders/variousinfofolder/interviewfolder/sfii_funamitsu/funamitsu3.jpg}{part 3}.

  \bibitem{mame_cps1_video}
  \textbf{"Driver vidéo du Mame CPS-1"} (mame source code),
  \href{https://github.com/mamedev/mame/blob/e070405df99e6a5997d5a64ecd62e7161c729a9d/src/mame/video/cps1.cpp#L269}{link},
  2008-04-11.

  \bibitem{mame_driver}
  \textbf{"Driver Mame CPS-1"} (mame source code),
  \href{https://github.com/mamedev/mame/blob/e070405df99e6a5997d5a64ecd62e7161c729a9d/src/mame/drivers/cps1.cpp#L567}{link},
  2008-04-11.

  \bibitem{mame_kabuki}
  \textbf{"Chiffrement du Kabuki z80"} (mame source code),
  \href{https://github.com/mamedev/historic-mame/blob/master/src/mame/machine/kabuki.c}{link},
  2008-04-11.


   \bibitem{cps0chipslist}
  \textbf{"Les 1er jeux Capcom FGPA"} (Jose Tejada),
  \href{https://github.com/jotego/jt_gng/blob/fb92e5ac0f72323638974034ad652649b6efafcb/README.md}{link},
  2020-08-05.

   \bibitem{h40}
  \textbf{"Le mode H40 de la Megadrive"}
  Les fréquences verticales et horizontales en mode H40 ne correspondent pas aux valeurs que l'on obtiendrait en injectant simplement la fréquence du dot-clock, le nombre de points et le nombre de lignes dans les formules.
  Cela s'explique par le fait que les concepteurs de la Genesis voulaient obtenir la même fréquence en modes H32 et H40 (59,92 Hz).
  Le dot-clock ralentit à 5,37 MHz pendant 28 points durant la période HBLANK, ce qui donne une VSYNC à 59,92 Hz et une HSYNC à 15 700 kHz (conversation avec Upsilandre).

   \bibitem{par}
  \textbf{"Dot clock rates"} (pineight.com),
  \href{https://pineight.com/mw/page/Dot_clock_rates.xhtml}{link}.

     \bibitem{ffdevinterview}
  \textbf{"L'interview des développeur de Final Fight"} (capcom.com),
  \href{https://game.capcom.com/cfn/sfv/column/132673?lang=en}{link},
  2019-02-08.

       \bibitem{sf2devinterview}
  \textbf{"L'interview des développeur Street Fighter II"} (capcom.com),
  \href{https://game.capcom.com/cfn/sfv/column/132595?lang=en}{link},
  2018-11-21.

     \bibitem{ar20160331}
  \textbf{"Rapport d'activité de Capcom : Akira Yasuda partie 1"} (capcom.com),
  \href{https://game.capcom.com/cfn/sfv/column/112429}{link},
  2016-03-31.

     \bibitem{ar20160404}
  \textbf{"Rapport d'activité de Capcom : Akira Yasuda partie 2"} (capcom.com),
  \href{https://game.capcom.com/cfn/sfv/column/112432}{link},
  2016-04-04.

     \bibitem{tgm198906}
  \textbf{"Capcom, un public captif"} (Robin Hogg \& Dominic Handy for The Games Machine, Issue \#19),
  \href{https://archive.org/details/the-games-machine-19/page/n23/mode/2up}{link},
  1989-06-01.

     \bibitem{gamest38}
  \textbf{"L'interview de Yoshiki Okamoto"} (Gamest Magazine \#38),
  \href{https://retrocdn.net/images/9/91/Gamest_JP_038.pdf}{link},
  1989-10-01.

     \bibitem{ffoverload}
  \textbf{"Final Fight arcade 2 joueurs"} (arronmunroe),
  \href{https://youtu.be/HyAHGHo22Og?t=1707}{link} (Use ',' and '.' to move frame by frame)
  2013-10-12.

    \bibitem{petitCRT}
  \textbf{"DL-0921 (CPS-B-21) Generation de signals"} (Lo\"{i}c Petit),
  \href{https://gitlab.com/loic.petit/cps2-reverse/-/blob/master/DLs/DL-0921/doc/video-signals.md}{link},
  2020-11-29.

    \bibitem{petitSecurity}
  \textbf{"DL-0921 (CPS-B-21) Schema de sécurité"} (Lo\"{i}c Petit),
  \href{https://gitlab.com/loic.petit/cps2-reverse/-/blob/master/DLs/DL-0921/doc/security-scheme.md}{link}.

    \bibitem{arcadeHackerCPS1}
  \textbf{"Capcom CPS1"} (Eduardo Cruz),
  \href{http://arcadehacker.blogspot.com/2015/04/capcom-cps1-part-1.html}{part 1},
  \href{http://arcadehacker.blogspot.com/2015/05/capcom-cps1-part-2.html}{part 2},
  \href{http://arcadehacker.blogspot.com/2015/06/capcom-cps1-part-3.html}{part 3},
  2015-04-16.
  
     \bibitem{arcadeHackerKabuki}
  \textbf{"Capcom Kabuki CPU"} (Eduardo Cruz),
  \href{ http://arcadehacker.blogspot.com/2014/11/capcom-kabuki-cpu-intro.html}{intro},
  \href{http://arcadehacker.blogspot.com/2014/11/capcom-kabuki-cpu-part-1.html}{part 1},
  \href{http://arcadehacker.blogspot.com/2014/11/capcom-kabuki-cpu-part-2.html}{part 2},
  \href{http://arcadehacker.blogspot.com/2014/11/capcom-kabuki-cpu-part-3.html}{part 3},
  \href{http://arcadehacker.blogspot.com/2014/11/capcom-kabuki-cpu-part-4.html}{part 4},
  \href{http://arcadehacker.blogspot.com/2014/11/capcom-kabuki-cpu-part-5.html}{part 5},
  2014-11-16.

  \bibitem{arcadeHackerCPS1Rev}
  \textbf{"CAPCOM CPS1 Reverse Engineering"} (Eduardo Cruz),
  \href{https://www.youtube.com/watch?v=IBZc__9sM28}{link},
  2015-06-15.

    \bibitem{arcadeHackerCPS1Desuicide}
  \textbf{"CPS1 Project Update"} (Eduardo Cruz),
  \href{http://arcadehacker.blogspot.com/2015/09/project-update.html}{link},
  2015-09-19.

  \bibitem{ieee20170630}
  \textbf{"Classement des puces : Motorola MC68000 Microprocessor"} (spectrum.ieee.org),
  \href{https://spectrum.ieee.org/tech-history/silicon-revolution/chip-hall-of-fame-motorola-mc68000-microprocessor}{link},
  2017-06-30.

 \bibitem{M68000fv}
  \textbf{"Prélecture des instructions sur le processeur Motorola 68000"} (Jorge Cwik),
  \href{http://pasti.fxatari.com/68kdocs/68kPrefetch.html}{link},
  2005.

    \bibitem{cps2rebirth}
  \textbf{"La renaissance du CPS-2 !!!!"} (cps2shock.retrogames.com),
  \href{https://web.archive.org/web/20060812042251/http://cps2shock.retrogames.com/wip.html}{link},
  2003-04-23.

    \bibitem{cps2_plain_ROM}
     \textbf{"Maintenant, nous avons une version déchiffrée dela ROM de SFZ"} (cps2shock.retrogames.com),
  \href{https://web.archive.org/web/20060812042426/http://cps2shock.retrogames.com/wipold.html}{link},
  2000-12-32.

    \bibitem{cps2_keys}
     \textbf{"Driver Mame CPS-2, keys (cps2crpt.c)"},
  \href{https://github.com/mamedev/historic-mame/blob/6f03755e212f33ae6e4681d56351bf3c44f20916/src/mame/machine/cps2crpt.c#L752}{link}.


     \bibitem{dalhsimGlitch}
  \textbf{"Le bug du Dhalsim invisible dans Street fighter 2 WW"} (youtube.com Error1),
  \href{https://www.youtube.com/watch?v=qEFPzcOK_uQ}{link},
  2010-09-22.

     \bibitem{otaquest}
  \textbf{"Mélanger les mondes par la musique : entretien avec la compositrice Yoko Shimomura"} (otaquest.com),
  \href{https://www.otaquest.com/yoko-shimomura-interview/}{link},
  2019-12-26.

     \bibitem{fmProgramming}
  \textbf{"Guide du programmeur pour le synthétiseur FM Yamaha YMF 262/OPL3"} (Vladimir Arnost),
  \href{https://www.fit.vutbr.cz/~arnost/opl/opl3.html}{link},
  2019-12-26.

     \bibitem{cpsBNumbers}
  \textbf{"CPS-B Number"} (tim for arcadecollection.com),
  \href{http://www.arcadecollecting.com/info/cps-b_numbers.html}{link}.

  
     \bibitem{tuhojgdv2}
  \textbf{"L'histoire méconnue des développeurs de jeux japonais, Volume 1 (Entretien : Koichi Yotsui)"} (John Szczepaniak),
  2015-11-04.

     \bibitem{gameMaestro4}
  \textbf{"Game Maestro \#4"}, \href{http://shmuplations.com/akiman/}{link}.

  \bibitem{sf2_oral_history}
  \textbf{"Street Fighter 2: L'histoire non écrite"} (Matt Leone),
  \href{https://www.polygon.com/a/street-fighter-2-oral-history/}{link}.
  2014-02-03.
  
    \bibitem{beep199010}
  \textbf{"BEEP ! Magazine Megadrive : Les femmes dans la création de jeux vidéo"} (translated shmuplations.com),
  \href{https://shmuplations.com/womenofgamedesign/}{link}.
  1990-10.

    \bibitem{strider_conversion}
  \textbf{"La conversion de Strider non terminée"} (Shoestring),
  \href{https://www.jammarcade.net/strider-conversion/}{link}.
  2016-02-17.

    \bibitem{csp1_phoenix}
  \textbf{"Comment Phoenixer une PCB CPS 2 ?"} (Joe Bagadonuts),
  \href{https://www.youtube.com/watch?v=HFj8Mkw_kog}{link}.
  2015-05-18.

 \bibitem{adpcm_specs}
  \textbf{"Algorithme ADPCM de Dialogic"} (Dialogic Corporation),
  \href{https://multimedia.cx/mirror/dialogic-adpcm.pdf}{link}.
  1988.

 \bibitem{sf2manual}
  \textbf{"Manuel de Street Fighter 2"} (Capcom Corporation),
  \href{https://www.gamesdatabase.org/Media/SYSTEM/Arcade/Manual/formated/Street_Fighter_II--_Champion_Edition_-_1992_-_Capcom.pdf}{link}.
  1992.

 \bibitem{sf2aiengine}
  \textbf{"Street Fighter 2: The moteur d'IA"} (Ben Torkington),
  \href{https://sf2platinum.wordpress.com/2017/01/20/the-ai-engine}{link}.
  2017-1-20

 \bibitem{x68000spritedoubler}
  \textbf{"Gestion des sprites sur le X68000"} (Koichi Yoshida),
  \href{https://yosshin4004-github-io.translate.goog/x68k/xsp/index.html?_x_tr_sl=ja&_x_tr_tl=en&_x_tr_hl=en-US}{link}.
  2021-02-25

 \bibitem{htmcc}
  \textbf{"Comment créer les personnages de combat de Capcom"} (Akiman, Kiki, Bengus),
  ISBN: 978-1772941364.
  2020-010-20

 \bibitem{akiman2003}
  \textbf{"Akiman, entretien de 2003 tiré de Capcom Design Works"} (Akiman, translated shmuplations),
  \href{http://shmuplations.com/akirayasuda/}{link}.
  2003

 \bibitem{YoshikiOkamotoTakashiNishiyama}
  \textbf{"Discussion entre les créateurs de Street Fighter et Fatal Fury : KOF"} (Yoshiki Okamoto and Takashi Nishiyama),
  \href{https://www.youtube.com/watch?v=uqRFod7nuHo&t}{link}.
  2021-08-09

 \bibitem{sf2completefiles}
  \textbf{"Street Fighter II Complete File"} (Capcom edition),
  ISBN: 978-4257090014.
  1992-11-15

 \bibitem{sf2musics}
  \textbf{"Shoryuken..! La musique de Street Fighter II"} (909originals),
  \href{https://909originals.com/2021/02/21/shoryuken-the-music-of-street-fighter-ii-how-yoko-shimomura-soundtracked-one-of-the-biggest-video-games-of-all-time/}{link}
  2021-21-02

 % \bibitem{sf2samples}
 %  \textbf{"The people who lent their voices to the characters of Street Fighter II - a thread"} (VGDensetsu),
 %  \href{https://twitter.com/vgdensetsu/status/1357040620343328771?lang=en}{link}
 %  2021-03-02

 \bibitem{x68000usage1}
  \textbf{"Le SDK du CPS-1 (CAT-A)"} (Akiman),
  \href{https://twitter.com/akiman7/status/1013648367837048833}{link}
  2018-07-01
  
   \bibitem{x68000usage2}
  \textbf{"Le SDK du CPS-1 (CAT-A) - Détails supplémentaires"} (Takenori Kimoto (a.k.a KimoKimo)),
  \href{https://twitter.com/KIMOKIMO_Club/status/1013817480496627712}{link}
  2018-07-02

   \bibitem{ohXarticle}
  \textbf{"Private View: \begin{CJK}{UTF8}{min}月刊電脳倶楽部 (GEKKAN DENNŌ CLUB)\end{CJK}"} (Ted Danson),
  \href{http://thecoter.ie/2015/06/25/gekkan-denno-club/}{link}
  2015-06-25
  
     \bibitem{MSM6295_datasheet}
  \textbf{"MSM6295 datasheet"} (by OKI),
  \href{https://fabiensanglard.net/sf2_sound_system/MSM6295.pdf}{link}
  
      \bibitem{smc70tech}
  \textbf{"Le Micro-ordinateur Sony SMC-70"} (by Ahm),
  \href{http://users.glitchwrks.com/~ahm/smc70/}{link}
  2011-05-19

      \bibitem{1991_retro}
  \textbf{"Capcom – Entretien rétrospectif"} (by https://shmuplations.com/),
  \href{https://shmuplations.com/capcom1991/}{link}
  1991

      \bibitem{sf2_music}
  \textbf{"Street Fighter II - L'interview de la bande son OST"} (Yoko Shimomura),
  \href{https://www.youtube.com/watch?v=HOxN8Dzv2sw}{link}
  2017-10-28
  
        \bibitem{sf2musicsecurity}
  \textbf{"Diggin' In The Carts : niveaux cachés"} (conversation Yoko Shimomura with Manami Matsumae),
  \href{https://www.youtube.com/watch?v=Y__usQbGA5M}{link}
  2014-09-23

        \bibitem{sf2platinium}
  \textbf{"Street Fighter II - Code source Platinum"} (Ben Torkington),
  \href{https://github.com/bentorkington/sf2ww}{link}
  2021-10-10

       \bibitem{9800lines}
  \textbf{"Les normes techniques japonaises : implications pour le commerce mondial et la compétitivité"} (John Mcintyre),
  ISBN: 978-1567200539.
  1997-02-28

       \bibitem{human68k_manual}
  \textbf{"Manuel du Human68k"} (gamesx.com),
\href{https://gamesx.com/wiki/doku.php?id=x68000:human68k_manual}{link}
  2019-08-27

       \bibitem{x68k_games_analysis}
  \textbf{"Le x68000 et la supériorité japonaise"} (upsilandre),
\href{https://www.gamopat-forum.com/t38282p210-le-x68000-et-la-superiorite-japonaise#3336620}{link}
  2020-12-04

       \bibitem{x68k_perfect_catalogue}
  \textbf{"X68000 Perfect Catalogue"} (by G-Walk),
ISBN: ISBN4867171018.
  2020-10-27

\end{thebibliography}
