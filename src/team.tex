\chapter{Peopleware} 

In its early years, Capcom organized its pipeline to produce multiple games in parallel. Three teams, named first, 2nd and 3rd Planning Rooms worked side by side. 

% Capcom always had multiple arcade projects worked on in parallel. Teams size varied but could grow rather big by US standards. A game like "Street Fighter 2" credits not less than thirty and one people.
As it grew and diversified, the gaming company restructured itself in 1988 to be made of two divisions. One dedicated to arcades and the other home consoles. 

After its restructuration, Capcom kept the same modus operantis where each team were siloed from each others with an internal organization favoring a strong sense of hierarchy. 



% Teams were not organized with static staff alike the famous Pixar way. Depending on the load, people could be assigned to work with a different Planner.

As layered as they where, operations where not set in stone and employees could climb the ladder quickly. 

A key figure in the production of "Street Fighter 2", Akira - Akiman - Yasuda, was hired on "Dyn Side Arms" as a SCROLL artist. Two years later, He was a Planner on Forgotten Worlds.

\begin{trivia}
The role of each person is clearly established. Try to checkout credits on \url{mobygames.com}. You will find OBJects designers and SCRoll designers there.
\end{trivia}

The sequel to "Street Fighter 1" was on Capcom schedule as early as 1988 when Capcom was still based in Osaka.

\begin{figure}[H]
\nbdraw{japan}
\caption*{JAPAN! Tadadadadada din din!}
\end{figure}








\subsection{Street Fighter '89}
Under the insistence of their US branch, Capcom started to work on a beat'em up, a genre reportedly very popular in the West at the time.

At first the plan was to recycle the I.P of "Street Fighter 1" by including Ken and Ryu as the main protagonists. The title of the game was to be "Street Fighter '89".

\begin{figure}[H]
\img{sf_89.png}
\caption*{Final Fight poster with SF2 title!}
\end{figure}

After receiving feedback from operators stating that players did not see the connection with "Street Fighter 1", Capcom renamed the game "Final Fight".

Production ran into problems when ROM shortage drove up prices, forcing the team to make a game on a GFX ROM budget half of Forgotten Worlds. Despite its modest budget, Final Fight became a world-wide hit.

Upon release, it sold 30,000 arcade units. It was the second most successful table arcade cabinet of the month in Japan. It went on to be the highest-grossing arcade game of 1990 in Japan as well as the second highest-grossing arcade game of 1991. 

This detour ended up having a profound impact on the actual sequel. 

\begin{q}{akiman\cite{gameMaestro4}}
  Because Final Fight was a hit, we were allowed to hire a lot more people than usual. 

  We were also allowed to use much more ROM chips per game. The budget was 16MB.
\end{q}






\section{Teams organization}
When the time finally came for Street Fighter 2, Capcom enthused the project with its two planners who had worked together on "Forgotten Worlds" and "Final Fight". By the time the game shipped, Akiman and "Akira - Nin - Nishitani" would be at the head of a 30+ person team.

\begin{trivia}
If the name "Akiman" may not have evoked much to players, "Nin" definitely should. It is the name associated with all high-scores and featured in innumerable "Final Fight" and "Street Fighter 2" screenshots, including these found in this book on page \pageref{nin_ff} and page \pageref{nin_sf2}.  
\end{trivia}

The project started in May 1990 and took ten months to complete. The game was released on February 1991.

\nbdraw{team}

\simg{0.85}{cabinet.jpg}




\section{Design}

\begin{figure}[H]
\img{sf2-orig1.png}
\caption*{Original characters}
\end{figure}

\begin{figure}[H]
\img{street-fighter-ii-original-island.jpg}
\caption*{The original design placed the competition on an island}
\end{figure}

This is gold \url{http://retro.land/street-fighter-ii/}



\begin{q}{akiman\cite{gameMaestro4}}
The previous game had only one protagonist, but now there were 8 to choose from. It was the first game with that kind of design. From the very beginning, I knew this was going to be a hit.
\end{q}


\begin{q}{Noritaka Funamitsu\cite{retro}}
  The first Street Fighter was not a success. But when everybody forgot about it, fighting games started to become popular inthe US. So we were asked to make a sequel.

  When we presented Final fight to the US, they asked us to make fighting games our priority. I felt uneasy. When we finished FF they told us we were wrong, that it would never succeed in the US. FF became a huge success. It sold more than 80,000 units in the US.
\end{q}



Barrels in FF are the same sprite as in SF2.


Capcom used three teams to work in parallel. After team two finished Forgotten Worlds (did not sell well\cite{ar20160331} and two years to develop into 1 meg rom chip) and Final Fight (huge hit), they started working on Street Fighter 2.

\begin{q}{akiman\cite{ar20160331}}
Back then the dev hierarchy was Planning > Character Design > Backgrounds.
\end{q}

DRWAING TEAM 2



\subsection{Design doc}

\begin{figure}[H]
\img{street-fighter-ii-logos.jpg}
\caption*{Street Fighter 2 logo research done by in-house graphic designer, Shoei Okano. Okano also ainmated Ryu.}
\end{figure}











\subsection{Anatomy guide}
\section{Musics}
\section{Sounds effects}




\section{Graphics}
Bla bla bla

\begin{q}{dlfrsilver, \url{https://eab.abime.net/showpost.php?p=1291841&postcount=39}}
the first CPS1 games were made graphically with a Sony SMC-70 computer.
The programming was done with PC-88jr computers.

From 1990, they made the games on the X68k computers, all the CPS1 game and all the CPS2 games were made on those, until 1995, where they ported the toolchain on Windows PC.
\end{q}






\img{sagat_character_drawings.jpg}\pagebreak

\begin{q}{Nishitani\cite{ffdevinterview}}
  
In order to make the best use of the capacity we had, we wrote the ROM's capacity on a board, and cut and paste the pixel characters on the board.
 
If there was space left on the board, then there was open capacity in the ROM. So, from there we started filling in the spaces, like a puzzle. One thing that happened that's kinda interesting, we saved making the ending for last, and by the time we got there we were all out of capacity. We were wondering what to do, when we found a board that had gone missing under a desk. We called it the "Miraculous Memory."


\end{q}

\img{rom_sheet.png}\pagebreak

\img{rom_sheet_dhalsim.jpg}\pagebreak

\img{rom_sheet_honda.jpg}\pagebreak

\img{rom_sheet_ryu.jpg}\pagebreak

\img{sf2_design_doc.jpg}\pagebreak



\begin{q}{Noritaka Funamizu, lead producer\cite{retro}}
  While our CPS-1 didn't offer the latest technology or CG, it was flexible, able to give the creators the possibility to use and modify data in all the ways.


  It could handle a large amount of different graphics data onscreen. It addressed many memory issues, which made our work easier. While Ryu was made in 8 Mbit, we were able to make Zangief 12 Mbit. On other boards, you have to make every character in 8 Mibt.
  \end{q}

\section{Programmers}






