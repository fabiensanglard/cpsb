\chapter{Personnalités}
\label{people}

De nombreuses personnes ont été mentionnées au fil de ce livre. Au début des années 90, la division arcade de Capcom était déjà importante, avec trois équipes travaillant sur des titres indépendants. Il peut être fastidieux de savoir qui a travaillé sur quoi. Voici un résumé.

\textbf{Kenzo Tsujimoto}, (\begin{CJK}{UTF8}{min}辻本憲三\end{CJK}) : A fondé Irem en 1974, une entreprise qui allait créer de nombreux jeux, dont les légendaires *R-Type* et *Kung-Fu Master*. Écarté pour des raisons financières après la sortie de *IPM Invaders* (1979), il fonde Capcom en 1983 et la mène vers un succès mondial. En 2022, il est toujours PDG de Capcom.

\textbf{Poo} (\textbf{Noritaka Funamizu}, \begin{CJK}{UTF8}{min}船水 紀孝\end{CJK}) : A rejoint Capcom en 1985 en tant que planneur sur des titres CPS-1 comme *Forgotten Worlds*, *U.N. Squadron*, *Dynasty Wars* et *1941: Counter Attack*. Il est également crédité dans les "Remerciements spéciaux" de pratiquement tous les succès de Capcom, y compris *Final Fight* et toutes les versions de *Street Fighter II*. Il est devenu plus tard producteur général et a travaillé sur de nombreux titres *Street Fighter*. En 2004, il quitte Capcom pour cofonder Crafts \& Meister.

\textbf{Akiman} (\textbf{Akira Yasuda}, \begin{CJK}{UTF8}{min}安田 朗\end{CJK}) : A rejoint Capcom en 1985. Jeune artiste, il s'occupe d'abord des décors dans *Hyper Dyne Side Arms*. Après avoir, dit-on, demandé une promotion lors d'une rencontre dans les toilettes, il devient planneur (centré sur l'aspect visuel) sur *Forgotten Worlds*, *Final Fight* et *Street Fighter II*. Il se consacre ensuite à l'illustration pour de nombreux jeux Capcom bien après l'ère du CPS-1. Il quitte Capcom en 2003 pour devenir artiste freelance.

\textbf{Nin} (\textbf{Akira Nishitani}, \begin{CJK}{UTF8}{min}西谷 亮\end{CJK}) : Rejoint Capcom en 1985. Son surnom "NiN" est bien connu car il apparaît dans les scores élevés de tous les jeux qu'il a planifiés. Parmi de nombreux titres, il est planneur (centré sur le gameplay) sur *Forgotten Worlds*, *Final Fight* et *Street Fighter II*. Il quitte Capcom en 1995 pour fonder Arika et produit la série *Street Fighter EX* pour son ancien employeur.

\textbf{Professor F} / \textbf{Arthur King} (\textbf{Tokuro Fujiwara}, \begin{CJK}{UTF8}{min}藤原 得郎\end{CJK}) : Rejoint Capcom en 1983 et planifie *Commando*, *Ghosts 'n Goblins* et *Bionic Commando*. Il devient directeur général en 1988. Après 13 ans chez Capcom, il quitte l'entreprise en 1996 pour fonder son propre studio, Whoopee Camp.

\textbf{Kouichi Yotsui} (\begin{CJK}{UTF8}{min}四井浩一\end{CJK}) : Rejoint Capcom en 1986 et planifie le tout premier jeu CPS-1, le spectaculaire *Strider*. Il quitte Capcom en 1990 pour travailler chez Takeru, puis Mitchell Corporation. En 2022, il travaille toujours en freelance.

\textbf{Yoshiki Okamoto} (\begin{CJK}{UTF8}{min}岡本 吉起\end{CJK}) : Rejoint Capcom en 1984 après avoir quitté Konami. Il est planneur sur des titres comme *Side Arms* et *Willow*, avant de superviser tout le développement arcade chez Capcom en tant que producteur. Il est notamment connu pour avoir recruté Akiman, ce qui aura un impact majeur sur la division arcade. Il quitte Capcom pour fonder Flagship, sa propre société de développement de jeux. Plus tard, il crée plusieurs jeux mobiles à succès comme *Dragon Hunter* et *Monster Strike*.

\textbf{Takashi Nishiyama} (\begin{CJK}{UTF8}{min}西山隆志\end{CJK}) : Rejoint Capcom en 1986, en provenance d'Irem où il a conçu *Kung-Fu Master*. Il est planneur sur plusieurs jeux antérieurs au CPS-1, notamment *Street Fighter 1*, pour lequel il invente le coup spécial "Hadouken". Il rejoint SNK en 1990, où il travaille sur *Fatal Fury* puis devient producteur sur plusieurs opus de *King of Fighters* et *Metal Slug*.

\textbf{Yoko Shimomura} (\begin{CJK}{UTF8}{min}下村 陽子\end{CJK}) : Rejoint Capcom en 1988 juste après avoir obtenu son diplôme du Osaka College of Music. Elle participe aux musiques de plus de seize jeux, débutant sur consoles avant de se tourner vers l'arcade. Elle est surtout connue pour avoir composé les musiques de *Street Fighter II* et *Final Fight*. Elle quitte Capcom en 1993 pour rejoindre Square, où elle travaille encore en 2022.

\textbf{Yoshihiro Sakaguchi} (\begin{CJK}{UTF8}{min}坂口 由洋\end{CJK}) : Rejoint Capcom en 1984. Il compose de la musique pour des titres consoles tels que *Mega Man* et *Mega Man 2*, ainsi que pour des jeux d'arcade comme *Street Fighter 1* et *Final Fight*. Il quitte Capcom au milieu des années 90.
