\chapter*{Avant-propos}

Il fut un temps où les passionnés de jeux vidéo ne pouvaient vivre les meilleures expériences que dans des lieux appelés "salles d’arcade".

Au début des années 90, les consoles de salon 16 bits comme la Super Nintendo, la Sega Genesis ou la NEC PC Engine gagnaient en puissance. Cependant, elles étaient loin d’égaler le matériel que l’on trouvait dans les "machines d’amusement" à pièces.

Surnommées "bornes à pièces", ces machines faisaient tourner des jeux remplis de sprites gigantesques couvrant tout l’écran, de couleurs éclatantes, de sons numérisés et de musiques de haute qualité. Ces machines jouaient clairement dans une autre catégorie.

Se rendre dans une salle d’arcade relevait de l’aventure. Il fallait rassembler de la monnaie, trouver un moyen de transport et étudier une carte papier. Certains faisaient du covoiturage, d’autres prenaient leur vélo. Les plus chanceux avaient une salle de jeux vidéo dans leur ville, d’autres jouaient dans un bar miteux, entourés d’adultes désabusés.

La durée de jeu dépendait directement du niveau du joueur. Les pièces étaient dépensées avec soin, souvent après avoir observé les techniques des autres. La seule certitude à la fin de l’expédition : on rentrait les poches vides.

Malgré tous ces obstacles, les connaisseurs ne pouvaient résister à l’appel. Des joueurs de tous âges et de toutes origines se retrouvaient au même endroit, mus par une passion commune.

Les rangées de bornes créaient un environnement extrêmement compétitif, où les éditeurs n’avaient que quelques secondes pour capter l’attention – et surtout les pièces – des joueurs. C’est à cette époque qu’une jeune entreprise nommée Capcom parvint à s’imposer, enchaînant les chefs-d’œuvre et devenant une icône.

L’histoire de Capcom et la genèse de Street Fighter II, Ghouls 'n Ghosts et Final Fight mériteraient leur place dans les livres d’histoire. Malheureusement, lorsque j’ai commencé mes recherches sur le sujet, j’ai trouvé peu d’informations pour satisfaire ma curiosité, et presque rien du point de vue de l’ingénierie.

La rivalité féroce entre les éditeurs imposait une discrétion extrême. Les artistes, programmeurs et designers n’étaient crédités que sous pseudonyme, pour éviter tout recrutement. Quant au matériel qui alimentait les jeux de Capcom, il n’a jamais été révélé officiellement, si ce n’est sous un nom de code : \textbf{CP-System}.

Ce livre tente de lever le voile sur cette plateforme mystérieuse. Il s’agit d’une lettre d’amour technique à la machine qui a permis le succès phénoménal de Capcom.

-- Fabien Sanglard\\
Lien occasionnel vers le passé

Sunnyvale, Californie\\
\monthyeardate\today\\
