\chapter{Annexe}

\section{Fabrication}

Ce livre a été rédigé pour l'essentiel sur un Lenovo X1 Carbon Gen 9 sous Ubuntu 22.04.  
Ce fut une expérience profondément agréable de travailler avec une machine aussi fiable et incroyablement rapide.

\img{neofetch.png}

Quelques incursions dans l'univers de Windows 10 ont eu lieu pour utiliser Adobe Photoshop, lorsque mes compétences sur Gimp atteignaient leurs limites.

\subsection{Outils}

Le code source a été synchronisé grâce à l'excellent Github. \LaTeX\ a été rédigé avec Sublime Text 4. Les dessins ont été réalisés avec Inkscape. Les captures d'écran des jeux ont été faites avec Mame. La compilation a été effectuée avec \texttt{pdflatex}.

Le système de compilation repose sur un programme personnalisé en Golang, capable de fonctionner avec une simple commande \icode{build.go}. Il faut 1m53 pour générer le PDF complet en mode publication (300 dpi). Un mode débogage incrémental (100 dpi) s'exécute en 10 secondes.

Le lecteur PDF variait selon la plateforme. Sous Linux, \texttt{evince} était utilisé, tandis que \texttt{SumatraPDF.exe} était préféré sous Windows. Ces deux visionneuses étaient non seulement extrêmement rapides, mais elles supportaient aussi le rechargement automatique — une fonction salvatrice.

Tous ces projets ont reçu de généreuses donations (lorsqu'ils les acceptaient) en remerciement pour leur service.

\subsection{Police de caractères}

La police utilisée pour le texte principal et tous les dessins est Roboto (10 pt). Le design de la couverture rend hommage à "The C Programming Language" de Dennis Ritchie et Brian Kernighan. La police de la couverture est Novarese.

\subsection{Motivation}

Pour rester motivé jusqu'à la publication, j'ai reçu une quantité copieuse de soutien moral de la part de Rudy le chat et de mon incroyable épouse Victoria.
